\documentclass[12pt]{scrartcl}

\usepackage[
  a4paper, mag=1000,
  left=2cm, right=1cm, top=2cm, bottom=2cm, headsep=0.7cm, footskip=1.27cm
]{geometry}

\usepackage[T2A]{fontenc}
\usepackage[utf8]{inputenc}
\usepackage[english,russian]{babel}
\usepackage{cmap}
\usepackage{amsmath}
\usepackage{tabularx}
\usepackage{graphicx}
\usepackage{array}
\IfFileExists{pscyr.sty}{\usepackage{pscyr}}{}
\usepackage[parfill]{parskip}
\usepackage{lastpage}
\usepackage{setspace} % single spacing between lines
\usepackage{blindtext} % for generated text - can remove
\usepackage{titlesec} % set header spacing
\setlength{\parindent}{15pt} % paragraph indent

\titlespacing{\section}{0pt}{\parskip}{-\parskip}
\titlespacing{\subsection}{0pt}{\parskip}{-\parskip}
\titlespacing{\subsubsection}{0pt}{\parskip}{-\parskip}

\usepackage[numbered]{bookmark}
\clubpenalty=10000
\widowpenalty=10000

\usepackage{fancybox,fancyhdr}
\pagestyle{fancy}
\fancyhf{}
\fancyhead[C]{\small{Олимпиадное программирование (средний уровень). Тренировка 04 \\ Летняя компьютерная школа ``КЭШ'', 6--26 августа 2016 года}}

%user-defined commands

\newcommand{\inputFile}{стандартный ввод}
\newcommand{\outputFile}{стандартный вывод}

\begin{document}

\singlespacing

\section*{Задача A. Убить всех человеков }

\begin{tabularx}{\textwidth}{l l X}
    Имя входного файла: & \texttt{\inputFile} \\
    Имя выходного файла: & \texttt{\outputFile} \\
    Ограничение по времени: & $2$ секунды \\
    Ограничение по памяти: & $256$ мегабайт \\
\end{tabularx}

\begin{figure}[h]
	\centering
    \includegraphics[width=0.6\linewidth]{Bander}
\end{figure}

У робота Бендера есть мечта~---~он очень хочет ``убить всех человеков'', но для этого ему необходима огромная куча денег. Поэтому он решил попытать удачу в самом крупном казино во Вселенной, расположенном на Марсе. Одним из способов сорвать большой куш является победа в необычной инопланетной игре, в которой необходимо производить сложные преобразования определенной последовательности чисел. Не сильно вникая в правила Бендер чудом дошел до последнего хода,на котором ему необходимо найти два числа в последовательности, которые дают наибольшее произведение. Напишите программу, которая поможет роботу добиться успеха. 

\subsection*{Формат входных данных}
Вводится сначала число $N$~---~количество чисел в последовательности ($2 \leq N \leq 100$). Далее записана сама последовательность: $N$ целых чисел, по модулю не превышающих $1000$.

\subsection*{Формат выходных данных}
Выведите два искомых числа в любом порядке. Если существует несколько различных пар чисел, дающих максимальное произведение, то выведите любую из них.

\subsection*{Примеры}

\texttt
{
	\begin{tabularx}{0.9\textwidth}{| X | X |}
       \hline
       \multicolumn{1}{|c|}{\inputFile} & \multicolumn{1}{c|}{\outputFile} \\ 
       \hline 
       \parbox[t]{\textheight}
       {
        9 \\ 
		3 5 1 7 9 0 9 -3 10 \\
       } &
       9 10 \\
       \hline
       \parbox[t]{\textheight}
       {
        3 \\
		-5 -300 -12 \\
       } &
		-300 -12 \\
       \hline
    \end{tabularx}
}

\newpage

\section*{Задача B. Нюхоскоп }

\begin{tabularx}{\textwidth}{l l X}
    Имя входного файла: & \texttt{\inputFile} \\
    Имя выходного файла: & \texttt{\outputFile} \\
    Ограничение по времени: & $1$ секунды \\
    Ограничение по памяти: & $256$ мегабайт \\
\end{tabularx}

\begin{figure}[h]
	\centering
    \includegraphics[width=0.5\linewidth]{Franc}
\end{figure}

Профессор Фарнсворт во время очередных своих исследований изобрел потрясающее устройство~---~нюхоскоп. При помощи него можно распознавать множество запахов на расстоянии нескольких световых лет. Запахи являются ``составными'', то есть различные ароматы, смешиваясь, образуют совершенно новый запах. Профессор хочет узнать, какие запахи он может синтезировать из того, который он почувствовал, используя нюхоскоп. Помогите ему написать программу, которая сделает это за него, ведь у профессора проблемы с памятью, и он может попросту забыть, что собирался делать, пока вычисляет запахи сам. 

\subsection*{Формат входных данных}
Вывести все представления запаха $N$ ($2 \leq N \leq 40$). в виде совокупности других запахов (одинаковые запахи при смешивании в определенном количестве так же формируют другой запах). Каждому запаху соответсвует одно положительное целое число. Перестановка запахов нового способа представления не даёт.

\subsection*{Формат выходных данных}
В каждой строке выводится одно из представлений~---~ряд запахов. В сумме ароматы разделяются знаком "+".

\subsection*{Примеры}

\texttt
{
	\begin{tabularx}{0.9\textwidth}{| X | X |}
       \hline
       \multicolumn{1}{|c|}{\inputFile} & \multicolumn{1}{c|}{\outputFile} \\ 
       \hline 
       \parbox[t]{\textheight}
       {
        4 \\ 
       } &
      \parbox[t]{\textheight}
       {
        1+1+1+1 \\
		1+2+1 \\
		1+3 \\
		2+2  \\ 
       } \\
       \hline
    \end{tabularx}
}

\newpage

\end{document}

\documentclass[12pt]{scrartcl}

\usepackage[
  a4paper, mag=1000,
  left=2cm, right=1cm, top=2cm, bottom=2cm, headsep=0.7cm, footskip=1.27cm
]{geometry}

\usepackage[T2A]{fontenc}
\usepackage[utf8]{inputenc}
\usepackage[english,russian]{babel}
\usepackage{cmap}
\usepackage{amsmath}
\usepackage{tabularx}
\usepackage{array}
\usepackage[parfill]{parskip}
\usepackage{lastpage}
\usepackage{setspace} % single spacing between lines
\usepackage{blindtext} % for generated text - can remove
\usepackage{titlesec} % set header spacing
\setlength{\parindent}{15pt} % paragraph indent

\titlespacing{\section}{0pt}{\parskip}{-\parskip}
\titlespacing{\subsection}{0pt}{\parskip}{-\parskip}
\titlespacing{\subsubsection}{0pt}{\parskip}{-\parskip}

\usepackage[numbered]{bookmark}
\clubpenalty=10000
\widowpenalty=10000

\usepackage{fancybox,fancyhdr}
\pagestyle{fancy}
\fancyhf{}
\fancyhead[C]{\small{Олимпиадное программирование (средний уровень). Перебор. Ребусы. \\ Летняя компьютерная школа ``КЭШ'', 6--26 августа 2016 года}}

%user-defined commands

\newcommand{\inputFile}{отсутствует}
\newcommand{\outputFile}{стандартный вывод}

\begin{document}

\singlespacing

\section*{Задача A. Ребус}

\begin{tabularx}{\textwidth}{l l X}
    Имя входного файла: & \texttt{\inputFile} \\
    Имя выходного файла: & \texttt{\outputFile} \\
    Ограничение по времени: & $2$ секунды \\
    Ограничение по памяти: & $256$ мегабайт \\
\end{tabularx}

Напишите программу, которая найдет все решения ребуса \texttt{ABCD}$+$\texttt{DCBA}$=$\texttt{CEEC}

Решением ребуса называется такая замена букв цифрами, что каждая буква заменяется цифрой от 1 до 6
(другие цифры в этом ребусе не допускаются) так, что написанное равенство оказывается верным.
Разным буквам соответствуют разные цифры.


\subsection*{Формат выходных данных}
Выведите каждое решение ребуса на отдельной строке.
Решение выводится в виде верного равенства, соответствующего описанному ребусу,
в котором все буквы заменены цифрами.

\subsection*{Пример (для другого ребуса)}

Если бы мы решали ребус \texttt{A}$+$\texttt{B}$=$\texttt{CD},
и допускались бы цифры от 1 до 9, то вывод мог бы быть таким: \\
\texttt{
3+9=12 \\
4+8=12 \\
4+9=13 \\
5+7=12 \\
5+8=13 \\
5+9=14 \\
6+7=13 \\
6+8=14 \\
6+9=15 \\
7+5=12 \\
7+6=13 \\
7+8=15 \\
7+9=16 \\
8+4=12 \\
8+5=13 \\
8+6=14 \\
8+7=15 \\
8+9=17 \\
9+3=12 \\
9+4=13 \\
9+5=14 \\
9+6=15 \\
9+7=16 \\
9+8=17 \\
}
\newpage

\section*{Задача B. Треугольники}

\begin{tabularx}{\textwidth}{l l X}
    Имя входного файла: & \texttt{\inputFile} \\
    Имя выходного файла: & \texttt{\outputFile} \\
    Ограничение по времени: & $2$ секунды \\
    Ограничение по памяти: & $256$ мегабайт \\
\end{tabularx}

На плоскости даны $N$ точек. Никакие две точки не совпадают. Найдите треугольник с вершинами в этих точках, имеющий наименьший возможный периметр.

\subsection*{Формат входных данных}

Сначала вводится число $N$~---~количество точек ($3 \leq N \leq 50$), а затем $N$ пар вещественных чисел, задающих координаты точек.

\subsection*{Формат выходных данных}

Выведите три числа~---~номера точек, которые должны быть вершинами треугольника, чтобы его периметр был минимален. Если решений несколько выведите любое из них.

\subsection*{Примеры}

\texttt{
    \begin{tabularx}{0.9\textwidth}{| X | X |}
        \hline
        \multicolumn{1}{|c|}{Ввод} & \multicolumn{1}{c|}{Вывод} \\ \hline
        \parbox[t]{\textheight}{
            5 \\
            0 0 \\
            1.3 0 \\
            -2 0.1 \\
            1 0 \\
            10 10 \\
        } & \parbox[t]{\textheight}{
            % Пример вывода. Каждая строчка заканчивается на \\ 
            1 2 4 \\
        } \\ \hline
    \end{tabularx}
}
\newpage

\end{document}

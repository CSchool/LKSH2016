\documentclass[10pt,a4paper]{article}
\usepackage[utf8]{inputenc}
\usepackage{amsmath}
\usepackage{amsfonts}
\usepackage{amssymb}
\usepackage{amsthm}
\usepackage{graphicx}
\usepackage[left=2cm,right=2cm]{geometry}

\usepackage[T2A]{fontenc}
\usepackage[english, russian]{babel}

\newtheorem{Def}{Def:}
\newtheorem{Lem}{Lem:}

\begin{document}

\section{Детерменированные конечные автоматы}

\begin{Def}
    Слово --- любая конечная последовательность символов.
\end{Def}

\begin{Def}
    Язык -- множество слов.
\end{Def}

\begin{Def}
    ДКА --- это следующий кортеж: $(\Sigma, q_0, Q, T, \delta)$, где
    \begin{enumerate}
        \item $\Sigma$ --- алфавит
        \item $Q$ --- это множество состояний
        \item $q_0$ --- стартовое состояние
        \item $T \subset Q$ --- множество терминальных состояний
        \item $\delta: Q \times \Sigma \rightarrow Q$
    \end{enumerate}
\end{Def}

\begin{Def}
$\varepsilon$ --- пустое слово
\end{Def}

Давайте обобщим функцию перехода следующим образом:

$$ \hat{\delta}(q, cs) = 
    \left\{
        \begin{matrix}
            q & cs = \varepsilon \\
            \hat{\delta}(q, c), s) & 
            c \in \Sigma, s \in \Sigma^*        \end{matrix}
    \right.
$$

Теперь будем говорить, что слово принимается автоматом если 
$\hat{\delta}(q_0, w) \in T$.

Как можно представить себе это: есть некоторый вычислитель -- автомат, он принимает или отвергает слова. Обработка слова идёт
посимвольно, каждый раз переходя на новый символ мы переходим в новое состояние, если в конце пришли в терминальное, то мы приняли слово.


Связи с этим можно нарисовать граф автомата. Вершины --- состояния, ребра -- переходы.

\subsection{Примеры автоматов:}

\begin{enumerate}
    \item Cлова, длина которых кратна 3м.
\end{enumerate}

\subsection{Задачки:}

\subsubsection{Двоичные числа кратные 2}

\subsubsection{Двоичные числа кратные 3}

\subsubsection{Двоичные числа кратные 6}

\subsubsection{Двочиные числа без 11}

\subsubsection{Любой конечный язык распознаётся автоматом}

\section{Пересечение, объединение, разность автоматов}

Давайте попытаемся перенести операции над множествами на автоматы,
т.е. хотим понять, что для всех этих операций ДКА существует.

\begin{Def}
    Пусть есть два ДКА: $A = (q_A, Q_A, T_A, \delta_A)$ и $B = (q_B, Q_B, T_B, \delta_B)$, тогда прямым произведением автоматов будет
    называться следующий автомат: $A \times B = ((q_A, q_B), Q_A \times Q_B, T_A \times T_B, \delta)$, где $\delta((a, b), c) = (\delta_A(a, c), \delta_B(b, c))$. 
\end{Def}

Поймем теперь, что такой автомат принимает пересечение языков автоматов.

\subsection{Задачи:}

\subsubsection{Объединение}

\subsubsection{Разница}

\subsubsection{Инверсия}

\subsubsection{Построить автомат чисел кратных 6 как произведение}

\section{Лемма о накачке}

Иногда нужно научиться показывать, что для какого-то языка нет автомата, для 
этого существует следующая лемма:

\begin{Lem}[О накачке]
    Пусть для языка $L$ существует ДКА, тогда:
     $$\exists n\colon \forall w \in L\colon |w| > n \Rightarrow 
         \exists x, y, z\colon 
             (w = xyz) \wedge 
             (|x| + |y| \leq n) \wedge 
             (|y| > 0) \wedge 
             \forall i > 0\colon xy^iz \in L $$
\end{Lem}

Т.е. для правильного языка, есть некоторая длина, что для любого слова больше неё
получаем, что его можно разбить на 3 части и середину можно повторять сколько угодно раз.

\begin{proof}
За $n$ возьмём количество состояний. Теперь расмотрим маршрут внутри автомата.
Т.к. слово длинное то должна быть петля по принципу Дирихле, значит теперь мы можем по ней накручиваться.
\end{proof}

\subsection{Задачи:}
    \subsubsection{$a^nb^n$}
    \subsubsection{Правильные скобочные последовательности}
    \subsubsection{Равное число 1 и 0}
    
\section{Динамика по автомату}

\end{document}
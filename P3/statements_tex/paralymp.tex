\documentclass[12pt]{scrartcl}

\usepackage[
  a4paper, mag=1000,
  left=2cm, right=1cm, top=2cm, bottom=2cm, headsep=0.7cm, footskip=1.27cm
]{geometry}

\usepackage[T2A]{fontenc}
\usepackage[utf8]{inputenc}
\usepackage[english,russian]{babel}
\usepackage{cmap}
\usepackage{amsmath}
\usepackage{tabularx}
\usepackage{array}
\usepackage[parfill]{parskip}
\usepackage{lastpage}
\usepackage{setspace} % single spacing between lines
\usepackage{blindtext} % for generated text - can remove
\usepackage{titlesec} % set header spacing
\setlength{\parindent}{15pt} % paragraph indent

\titlespacing{\section}{0pt}{\parskip}{-\parskip}
\titlespacing{\subsection}{0pt}{\parskip}{-\parskip}
\titlespacing{\subsubsection}{0pt}{\parskip}{-\parskip}

\usepackage[numbered]{bookmark}
\clubpenalty=10000
\widowpenalty=10000

\usepackage{fancybox,fancyhdr}
\pagestyle{fancy}
\fancyhf{}
\fancyhead[C]{\small{Олимпиадное программирование (высокий уровень). Паралимиада \\ Летняя компьютерная школа ``КЭШ'', 17 августа 2016 года}}

%user-defined commands

\newcommand{\inputFile}{стандартный ввод}
\newcommand{\outputFile}{стандартный вывод}

\begin{document}

\singlespacing

\section*{Задача A. Потеряв контроль}

\begin{tabularx}{\textwidth}{l l X}
    Имя входного файла: & \texttt{\inputFile} \\
    Имя выходного файла: & \texttt{\outputFile} \\
    Ограничение по времени: & $2$ секунды \\
    Ограничение по памяти: & $256$ мегабайт \\
\end{tabularx}

Вам дан полный неориентированный граф. Для каждой пары вершин задана длина ребра, соединяющего их.
Найдите кратчайшие пути между каждой парой вершин, и выведите длину самого длинного из этих путей.

\subsection*{Формат входных данных}
Первая строка входных данных содержит целое число $N$ ($3 \le N \le 10$).

Следующие $N$ строк содержат $N$ целых чисел, разделенных пробелами, каждая.
$j$-ое число в $i$-ой строке $a_{ij}$ задает длину ребра, соединяющего вершины $i$ и $j$.
$a_{ij} = a_{ji}, a_{ii} = 0, 1 \le a_{ij} \le 100$ для $i \ne j$.

\subsection*{Формат выходных данных}
Выведите наибольшую длину кратчайшего пути между какой-либо парой вершин в графе.

\subsection*{Примеры}

\texttt{
    \begin{tabularx}{0.9\textwidth}{| X | X |}
        \hline
        \multicolumn{1}{|c|}{\inputFile} & \multicolumn{1}{c|}{\outputFile} \\ \hline
        \parbox[t]{\textheight}{
3 \\
0 1 1 \\
1 0 4 \\
1 4 0 \\
        } & \parbox[t]{\textheight}{
            2 \\
        } \\ \hline
        \parbox[t]{\textheight}{
4 \\
0 1 2 3 \\
1 0 4 5 \\
2 4 0 6 \\
3 5 6 0 \\
        } & \parbox[t]{\textheight}{
            5 \\
        } \\ \hline
    \end{tabularx}
}

\subsection*{Примечание}
У вас заканчиваются ключевые слова, поэтому ваша программа не может содержать их как подстроку (в любом регистре):

\texttt{define, do, for, foreach, while, repeat, until, if, then, else, elif, elsif, elseif, case, switch}


\newpage



\section*{Задача B. Вода в чайнике}

\begin{tabularx}{\textwidth}{l l X}
    Имя входного файла: & \texttt{\inputFile} \\
    Имя выходного файла: & \texttt{\outputFile} \\
    Ограничение по времени: & $2$ секунды \\
    Ограничение по памяти: & $256$ мегабайт \\
\end{tabularx}

У Маргариты сегодня день рождения и она пригласила своих друзей в гости на праздник.
Она решила угостить всех своим самым лучшим чаем, для чего заблаговременно поставила чайник.
Будем считать, что чайник --- это цилиндр с радиусом основания $r$ и высотой $h$.
Так как Маргарита работает программистом, она может себе позволить самую современную технику.
Поэтому в её чайник встроено самое новое реле. Вообще говоря реле --- электронное устройство (ключ),
предназначенное для замыкания или размыкания электрической цепи при заданных изменениях электрических
или неэлектрических входных воздействий. В чайнике Маргариты специальное сверхточное реле, которое
выключает чайник, как только вода начинает кипеть. Известно, что чайник потребляет мощность $p$.
Какую температуру будет иметь вода, когда реле выключит чайник?

\subsection*{Формат входных данных}
В единственной строке находятся три целых числа $r$, $h$, $p$ ($1 \le r, h \le 100$, $1 \le p \le 10^5$)
--- радиус, высота и мощность чайника.

\subsection*{Формат выходных данных}
Выведите одно целое число --- ответ на задачу.

\subsection*{Примеры}

\texttt{
    \begin{tabularx}{0.9\textwidth}{| X | X |}
        \hline
        \multicolumn{1}{|c|}{\inputFile} & \multicolumn{1}{c|}{\outputFile} \\ \hline
        \parbox[t]{\textheight}{
            10 20 50 \\
        } & \parbox[t]{\textheight}{
            100 \\
        } \\ \hline
    \end{tabularx}
}
\newpage


\section*{Задача C. Игра}

\begin{tabularx}{\textwidth}{l l X}
    Имя входного файла: & \texttt{\inputFile} \\
    Имя выходного файла: & \texttt{\outputFile} \\
    Ограничение по времени: & $2$ секунды \\
    Ограничение по памяти: & $256$ мегабайт \\
\end{tabularx}

Лиза и Дима играют в интересную игру. Дима придумывает массив длины $n$ и прячет его
от Лизы. Ему очень нравится число $x$, поэтому как минимум $80\%$ элементов его массива равны $x$.
Лиза должна угадать любимое число Димы (то есть $x$). Для этого она может спрашивать, какое число стоит
на $i$-ом месте массива. Ваша задача --- помочь ей угадать, какое число любит Дима.

\subsection*{Формат входных данных}
Это интерактивная задача. В процессе тестирования ваша программа будет с использованием
стандартных потоков ввода/вывода взаимодействовать с программой жюри.

Ваша программа должна следовать следующему протоколу:

\begin{itemize}
    \item В начале ваша программа должна считать одно целое число $n$ ($1 \le n \le 10^5$).
    \item Для запроса элемента на $i$-ой позиции ваша программа должна вывести в стандартный поток
          вывода запрос в формате ``\texttt{? i}''. Обратите внимание, что должно выполняться
          ограничение $1 \le i \le n$. После этого ваша программа должна считать одно целое число
          $a_i$ ($1 \le a_i \le 10^9$) --- элемент массива Димы на $i$-ой позиции.
    \item Когда ваша программа найдёт любимое число Димы $x$, она должна его вывести в формате ``\texttt{+ x}''
          и завершить работу.
\end{itemize}

Ваша программа должна сделать не более 100 запросов.

Гарантируется, что хотя бы $80\%$ элементов загаданного массива --- это одно и то же число.

Запросы вашей программы должны завершаться переводом строки и сбросом буфера потока вывода. Для этого 
используйте \texttt{flush(output)} в Pascal; \texttt{fflush(stdout)} или \texttt{cout.flush()} в C/C++;
\texttt{sys.stdout.flush()} в Python.
\newpage



\section*{Задача D. A + B}

\begin{tabularx}{\textwidth}{l l X}
    Имя входного файла: & \texttt{\inputFile} \\
    Имя выходного файла: & \texttt{\outputFile} \\
    Ограничение по времени: & $2$ секунды \\
    Ограничение по памяти: & $256$ мегабайт \\
\end{tabularx}

Внимание! Идея задачи взята с пробного тура
школьной олимпиады по программированию 2003 года посёлка Ноябрьский.

\subsection*{Формат входных данных}
В единственной строке находятся два целых числа $A$ и $B$ ($1 \le A, B \le 10^3$).

\subsection*{Формат выходных данных}
Выведите одно целое число --- ответ на задачу.

\subsection*{Примеры}

\texttt{
    \begin{tabularx}{0.9\textwidth}{| X | X |}
        \hline
        \multicolumn{1}{|c|}{\inputFile} & \multicolumn{1}{c|}{\outputFile} \\ \hline
        \parbox[t]{\textheight}{
            2 5 \\
        } & \parbox[t]{\textheight}{
            10 \\
        } \\ \hline
    \end{tabularx}
}
\newpage


\section*{Задача E. Полдтмирнсоань}

\begin{tabularx}{\textwidth}{l l X}
    Имя входного файла: & \texttt{\inputFile} \\
    Имя выходного файла: & \texttt{\outputFile} \\
    Ограничение по времени: & $2$ секунды \\
    Ограничение по памяти: & $256$ мегабайт \\
\end{tabularx}

Нмноипам, что пролонимдам нсвыаазтея срокта, каоротя чтсиеатя оанкдовио как слева нврпаао, так и срвпаа навлео. Напмерир, пиалдмрманои явсялтюя сткори \texttt{abba} и \texttt{mdaam}.

Для пзьолоониврй сортки s вдевем опцраиею динееля палопом, обмеонучзааю $half(s)$.
Зиннчеае $half(s)$ оеяптедсреля сещмиюлдуи пилвмаари:

\begin{itemize}
    \item Если $s$ не яялствея паоомлрдним, то знчаение $hlaf(s)$ не оерпнеледо;
    \item Елси $s$ иеемт длину 1, то зичаенне $hlaf(s)$ ткаже не оерелденпо;
    \item Если $s$ явестляя пордмалноим чноетй днилы $2m$, то $hlaf(s)$ это сортка, сщотаясоя из пырвех $m$ слоомивв срткои $s$.
    \item Если $s$ яяелсвтя поирдманолм ннтчеоей днилы $2m + 1$, бльшеой 1, то $hlaf(s)$ это соктра, сощаяотся из пеыврх $m + 1$ силоомвв сркоти $s$.
\end{itemize}

Нмаерпир, зеничная $hlaf($\texttt{ioimaartnfcs}$)$ и $hlaf($\texttt{i}$)$ не опенлредеы,
$half($\texttt{abba}$) = $\texttt{ab}, \\ $hlaf($\texttt{madam}$) = $\texttt{mad}.
Пьмрлдононаитсю стркои $s$ буедм назтваыь миньаамлское число раз, которое монжо пиеимртнь к сткоре $s$ опаерицю делеиня паополм, чбтоы реьултазт был оерпеедлн.
Напмриер, потмснднлаироь сортк \texttt{ifcmaotnris} и \texttt{i} рвана 0,
так как к ним ньлзея пмтииенрь оциепарю диенлея поопалм джае один раз.
Потрдлсианонмь сротк \texttt{abba} и \texttt{mdaam} рнава 1,
а патнснордолимь сротки \texttt{ttototoott} ранва 3,
плкокоьсу оераицпя длеения палопом пнимемриа к ней три рзаа:
\texttt{totottotot} $\rightarrow$ \texttt{toott} $\rightarrow$ \texttt{tot} $\rightarrow$ \texttt{to}.

Зданаа ноотаркея сротка $s$. Нмоедиохбо витисчыль палтномосдинрь зндоаанй сортки.

\subsection*{Формат входных данных}
Пеавря скртоа соежидрт непутсую сокрту $s$, сщуотсяою из сторчынх букв латиокнсго афватила. Ее днлиа не прохводисет $10^5$ смооливв.

\subsection*{Формат выходных данных}
Вдвтиеые в кчеастве овтета псоинлмнратодь стокри, заоадннй во вохыднх днаынх.

\subsection*{Примеры}

\texttt{
    \begin{tabularx}{0.9\textwidth}{| X | X |}
        \hline
        \multicolumn{1}{|c|}{\inputFile} & \multicolumn{1}{c|}{\outputFile} \\ \hline
        \parbox[t]{\textheight}{
           informatics \\
        } & \parbox[t]{\textheight}{
            0 \\
        } \\ \hline
        \parbox[t]{\textheight}{
           madam \\
        } & \parbox[t]{\textheight}{
            1 \\
        } \\ \hline
        \parbox[t]{\textheight}{
           totottotot \\
        } & \parbox[t]{\textheight}{
            3 \\
        } \\ \hline
    \end{tabularx}
}
\newpage


\section*{Задача F. Игра в тесты}

\begin{tabularx}{\textwidth}{l l X}
    Имя входного файла: & \texttt{-{}-} \\
    Имя выходного файла: & \texttt{-{}-} \\
    Ограничение по времени: & --- \\
    Ограничение по памяти: & --- \\
\end{tabularx}

Разработайте систему тестов для задачи ``Игра в числа'' из первой тренировки.
Вам нужно сделать несколько тестов (от 1 до 20), удовлетворяющих условию задачи.
Тесты необходимо сохранить в файлах \texttt{001.dat},
\texttt{002.dat}, \texttt{003.dat} и т.д.
На проверку вам необходимо сдать архив в формате zip или tar.gz (настоятельно рекомендуется использовать tar.gz).
Внутри этого архива должен быть каталог с именем \texttt{tests}.
Внутри этого каталога должно быть не более 20 тестов.
Все тесты должны строго соответствовать формату входных данных, описанных в условии задачи.
Сданный файл получает OK, если он имеет правильный формат и структуру, содержит от 1 до 20 корректных
тестов и все правильные решения проходят все тесты,
а все неправильные решения не проходят хотя бы один тест из числа предложенных вами.
Для создания zip-архивов рекомендуется использовать архиваторы 7-zip или WinRAR.
Использовать встроенные средства Windows для создания zip-архивов нельзя,
так как они создают некорректные архивы.

\newpage

\end{document}

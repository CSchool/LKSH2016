\documentclass[12pt]{scrartcl}

\usepackage[
  a4paper, mag=1000,
  left=2cm, right=1cm, top=2cm, bottom=2cm, headsep=0.7cm, footskip=1.27cm
]{geometry}

\usepackage[T2A]{fontenc}
\usepackage[utf8]{inputenc}
\usepackage[english,russian]{babel}
\usepackage{cmap}
\usepackage{amsmath}
\usepackage{tabularx}
\usepackage{array}
\usepackage[parfill]{parskip}
\usepackage{lastpage}
\usepackage{setspace} % single spacing between lines
\usepackage{blindtext} % for generated text - can remove
\usepackage{titlesec} % set header spacing
\setlength{\parindent}{15pt} % paragraph indent

\titlespacing{\section}{0pt}{\parskip}{-\parskip}
\titlespacing{\subsection}{0pt}{\parskip}{-\parskip}
\titlespacing{\subsubsection}{0pt}{\parskip}{-\parskip}

\usepackage[numbered]{bookmark}
\clubpenalty=10000
\widowpenalty=10000

\usepackage{fancybox,fancyhdr}
\pagestyle{fancy}
\fancyhf{}
\fancyhead[C]{\small{Олимпиадное программирование (высокий уровень). Тренировка 04 \\ Летняя компьютерная школа ``КЭШ'', 6--26 августа 2016 года}}

%user-defined commands

\newcommand{\inputFile}{стандартный ввод}
\newcommand{\outputFile}{стандартный вывод}

\begin{document}

\singlespacing

\section*{Задача A. Ход конём}

\begin{tabularx}{\textwidth}{l l X}
    Имя входного файла: & \texttt{\inputFile} \\
    Имя выходного файла: & \texttt{\outputFile} \\
    Ограничение по времени: & $2$ секунды \\
    Ограничение по памяти: & $256$ мегабайт \\
\end{tabularx}

Шахматная ассоциация решила оснастить всех своих сотрудников такими телефонными номерами,
которые бы набирались на кнопочном телефоне ходом коня.
Например, ходом коня набирается телефон \texttt{340-4927}.
При этом телефонный номер не может начинаться ни с цифры 0, ни с цифры 8.

Клавиатура телефона выглядит так: \\
\texttt{
    \begin{tabular}{l l l}
        7 & 8 & 9 \\
        4 & 5 & 6 \\
        1 & 2 & 3 \\
          & 0 &   \\
    \end{tabular}
}


Напишите программу, определяющую количество телефонных номеров длины $N$, набираемых ходом коня.

\subsection*{Формат входных данных}
На входе записано целое число $N$ ($1 \le N \le 50$)


\subsection*{Формат выходных данных}
Выведите искомое количество телефонных номеров.

\subsection*{Примеры}

\texttt{
    \begin{tabularx}{0.9\textwidth}{| X | X |}
        \hline
        \multicolumn{1}{|c|}{\inputFile} & \multicolumn{1}{c|}{\outputFile} \\ \hline
        \parbox[t]{\textheight}{
            2 \\
        } & \parbox[t]{\textheight}{
            16 \\
        } \\ \hline
    \end{tabularx}
}
\newpage



\section*{Задача B. Возрастающая подпоследовательность}

\begin{tabularx}{\textwidth}{l l X}
    Имя входного файла: & \texttt{\inputFile} \\
    Имя выходного файла: & \texttt{\outputFile} \\
    Ограничение по времени: & $5$ секунд \\
    Ограничение по памяти: & $256$ мегабайт \\
\end{tabularx}

Даны $N$ целых чисел $X_1, X_2, \ldots, X_N$.
Требуется вычеркнуть из них минимальное количество чисел так, чтобы оставшиеся шли в порядке возрастания.

\subsection*{Формат входных данных}
В первой строке находится число $N$ ($1 \le N \le 10000$).
В следующей строке --- $N$ натуральных чисел через пробел, числа не превосходят $60000$.

\subsection*{Формат выходных данных}
В первой строке выводится количество невычеркнутых чисел,
во второй --- сами невычеркнутые числа через пробел в исходном порядке. Если вариантов несколько, вывести любой.

\subsection*{Примеры}

\texttt{
    \begin{tabularx}{0.9\textwidth}{| X | X |}
        \hline
        \multicolumn{1}{|c|}{\inputFile} & \multicolumn{1}{c|}{\outputFile} \\ \hline
        \parbox[t]{\textheight}{
            6 \\
            2 5 3 4 6 1 \\
        } & \parbox[t]{\textheight}{
            4 \\
            2 3 4 6 \\
        } \\ \hline
    \end{tabularx}
}
\newpage
\end{document}

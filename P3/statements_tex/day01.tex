\documentclass[11pt]{scrartcl}

\usepackage[
  a4paper, mag=1000,
  left=2cm, right=1cm, top=2cm, bottom=2cm, headsep=0.7cm, footskip=1.27cm
]{geometry}

\usepackage[T2A]{fontenc}
\usepackage[utf8]{inputenc}
\usepackage[english,russian]{babel}
\usepackage{cmap}
\usepackage{amsmath}
\usepackage{tabularx}
\usepackage{array}
\usepackage[parfill]{parskip}
\usepackage{tabularx}
\usepackage{lastpage}

\usepackage[numbered]{bookmark}
\clubpenalty=10000
\widowpenalty=10000

\usepackage{fancybox,fancyhdr}
\pagestyle{fancy}
\fancyhf{}
\fancyhead[C]{\small{Олимпиадное программирование (высокий уровень). Тренировка 01 \\ Летняя компьютерная школа ``КЭШ'', 6--26 августа 2016 года}}

\begin{document}

\section*{Задача A. Шашечная доска}

\begin{tabularx}{\textwidth}{l l X}
    Имя входного файла: & \texttt{стандартный ввод} \\
    Имя выходного файла: & \texttt{стандартный вывод} \\
    Ограничение по времени: & $2$ секунды \\
    Ограничение по памяти: & $256$ мегабайт \\
\end{tabularx}

В каждой клетке шахматной доски $8 \times 8$ в произвольном порядке находится шашка одного из цветов: белая, чёрная, красная или зелёная.

\begin{tabular}{| l | l | l | l | l | l | l | l |}
    \hline
    2 & 1 & 1 & 0 & 3 & 0 & 3 & 1 \\ \hline
    0 & 1 & 1 & 3 & 3 & 1 & 0 & 0 \\ \hline
    1 & 2 & 1 & 3 & 1 & 0 & 1 & 2 \\ \hline
    1 & 1 & 1 & 1 & 2 & 2 & 1 & 0 \\ \hline
    1 & 1 & 1 & 0 & 1 & 2 & 1 & 2 \\ \hline
    0 & 1 & 0 & 1 & 1 & 2 & 1 & 1 \\ \hline
    0 & 0 & 0 & 0 & 0 & 0 & 0 & 0 \\ \hline
    1 & 1 & 1 & 2 & 2 & 2 & 3 & 3 \\ \hline
\end{tabular}

0 --- цвет и местоположение ЧЁРНОЙ шашки \\
1 --- цвет и местоположение БЕЛОЙ шашки \\
2 --- цвет и местоположение КРАСНОЙ шашки \\
3 --- цвет и местоположение ЗЕЛЁНОЙ шашки \\


\subsection*{Формат входных данных}
Данные о шашках записаны построчно и без пробелов в строке и между строками.

\subsection*{Формат выходных данных}
Составить программу, подсчитывающую количество шашек каждого цвета и выводящую результат в виде:
\begin{itemize}
\item данных о местоположении красных шашек (в остальных местах вывести знак \texttt{-})
\item пустой строки
\item количестве черных, белых, красных и зеленых шашек через пробел
\end{itemize}
Если шашки какого-либо цвета отсутствуют на доске, то вывести в файл сообщение \texttt{BAD INPUT LIST}.

\subsection*{Примеры}

\texttt{
    \begin{tabularx}{\textwidth}{| X | X |}
        \hline
        стандартный ввод & стандартный вывод \\ \hline
        \parbox[t]{\textheight}{
21103031 \\
01133100 \\
12131012 \\
11112210 \\
11101212 \\
01011211 \\
00000000 \\
11122233 \\
        } & \parbox[t]{\textheight}{
2-{}-{}-{}-{}-{}-{}- \\
-{}-{}-{}-{}-{}-{}-{}- \\
-2-{}-{}-{}-{}-2 \\
-{}-{}-{}-22-{}- \\
-{}-{}-{}-{}-2-2 \\
-{}-{}-{}-{}-2-{}- \\
-{}-{}-{}-{}-{}-{}-{}- \\
-{}-{}-222-{}- \\
\\
18 28 11 7 \\
        } \\ \hline
        \parbox[t]{\textheight}{
21111231 \\
11121333 \\
11131111 \\
22332233 \\
11132111 \\
22221121 \\
11113111 \\
12111121 \\
        } & \parbox[t]{\textheight}{
            BAD INPUT LIST \\
        } \\ \hline
    \end{tabularx}
}
\newpage


\section*{Задача B. Игра в числа}

\begin{tabularx}{\textwidth}{l l X}
    Имя входного файла: & \texttt{стандартный ввод} \\
    Имя выходного файла: & \texttt{стандартный вывод} \\
    Ограничение по времени: & $2$ секунды \\
    Ограничение по памяти: & $256$ мегабайт \\
\end{tabularx}

Вася очень любит играть в числа. Для этой игры нужна колода из $N$ различных целых чисел от 1 до $N$.
Эдик (который тоже очень любит эту игру) только что достал новую колоду.
Вася говорит, что в колоде недостаёт ровно двух чисел.
Зная $N$ (количество чисел, которое должно быть в колоде) и $S$ (сумму имеющихся чисел),
определите, не ошибся ли Вася, и если не ошибся, предложите, каких именно чисел недостаёт.

\subsection*{Формат входных данных}
В первой строке входных данных находятся два целых числа: $N$ ($3 \le N \le 10$)
и $S$ (от нуля до суммы чисел полной колоды).

\subsection*{Формат выходных данных}
Если могли пропасть ровно два числа, в первой строке напишите \texttt{yes},
во второй строке приведите возможный вариант --- два числа через пробел.
Если Вася ошибся, в единственной строке напишите \texttt{no}.

\subsection*{Примеры}

\texttt{
    \begin{tabularx}{\textwidth}{| X | X |}
        \hline
        стандартный ввод & стандартный вывод \\ \hline
        \parbox[t]{\textheight}{
            6 13 \\
        } & \parbox[t]{\textheight}{
            yes \\
            3 5 \\
        } \\ \hline
        \parbox[t]{\textheight}{
            4 9 \\
        } & \parbox[t]{\textheight}{
            no \\
        } \\ \hline
    \end{tabularx}
}
\newpage

\section*{Задача C. Разноцветная башня}

\begin{tabularx}{\textwidth}{l l X}
    Имя входного файла: & \texttt{стандартный ввод} \\
    Имя выходного файла: & \texttt{стандартный вывод} \\
    Ограничение по времени: & $2$ секунды \\
    Ограничение по памяти: & $256$ мегабайт \\
\end{tabularx}

Сережа играет с разноцветными кубиками --- строит из них башню.
Каждый кубик кроме самого большого он кладет на больший его. Самый большой кубик он ставит на пол в своей комнате.

Разумеется, при построении башни Сережа может использовать не все кубики.
Однако он хочет, чтобы получившаяся башня была как можно более высокой.
Для этого он хочет использовать при ее построении все кубики. К счастью, среди кубиков нет двух, имеющих одинаковый размер.

Кроме размера, каждый кубик характеризуется цветом --- красным, синим или зеленым.
Сережа хочет узнать, какая площадь поверхности построенной им башни будет окрашена в каждый из цветов.
Напишите программу, которая вычисляет ответ на сережин вопрос по описанию набора кубиков. 

\subsection*{Формат входных данных}
В первой строке входных данных находится целое число $N$ --- количество кубиков ($1 \le N \le 5 \cdot 10^4$).

Каждая из последующих n строк содержит описание одного кубика.
Описание $i$-го кубика состоит из целого числа $L_i$ $(1 \le L_i \le 10^7)$ ---
длины его ребра и символа (\texttt{R}, \texttt{G}, \texttt{B}), обозначающего его цвет.
Длины всех кубиков различны.

\subsection*{Формат выходных данных}
В выходной файл выведите для каждого цвета площадь поверхности башни, имеющей такой цвет.
Следуйте формату выходных данных, приведенному в примерах. 

\subsection*{Примеры}

\texttt{
    \begin{tabularx}{\textwidth}{| X | X |}
        \hline
        стандартный ввод & стандартный вывод \\ \hline
        \parbox[t]{\textheight}{
3   \\
1 R \\
2 G \\
3 B \\
        } & \parbox[t]{\textheight}{
R - 5  \\
G - 19 \\
B - 41 \\
        } \\ \hline
    \end{tabularx}
}
\newpage

\section*{Задача D. Про строки}

\begin{tabularx}{\textwidth}{l l X}
    Имя входного файла: & \texttt{стандартный ввод} \\
    Имя выходного файла: & \texttt{стандартный вывод} \\
    Ограничение по времени: & $2$ секунды \\
    Ограничение по памяти: & $256$ мегабайт \\
\end{tabularx}

Неоднородностью строки $S$ назовем разность между числом вхождений в нее самого
часто встречающегося символа и самого редко встречающегося символа. 
Например, для строки \texttt{aaabbc} неоднородность равна двум, так как наиболее часто
встречающийся символ \texttt{a} входит в нее три раза, а наиболее редко встречающийся символ \texttt{c} --- один раз.
Напишите программу, вычисляющую неоднородность заданной строки.

\subsection*{Формат входных данных}
Первая строка содержит непустую строку $S$ длиной не более 100 символов.
Строка $S$ состоит только из строчных букв латинского алфавита. 

\subsection*{Формат выходных данных}
Выведите ответ на задачу.

\subsection*{Примеры}

\texttt{
    \begin{tabularx}{\textwidth}{| X | X |}
        \hline
        стандартный ввод & стандартный вывод \\ \hline
        \parbox[t]{\textheight}{
            aaabbc \\
        } & \parbox[t]{\textheight}{
            2
        } \\ \hline
        \parbox[t]{\textheight}{
            abba \\
        } & \parbox[t]{\textheight}{
            0
        } \\ \hline
    \end{tabularx}
}
\newpage


\end{document}

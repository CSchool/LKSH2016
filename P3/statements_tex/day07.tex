\documentclass[12pt]{scrartcl}

\usepackage[
  a4paper, mag=1000,
  left=2cm, right=1cm, top=2cm, bottom=2cm, headsep=0.7cm, footskip=1.27cm
]{geometry}

\usepackage[T2A]{fontenc}
\usepackage[utf8]{inputenc}
\usepackage[english,russian]{babel}
\usepackage{cmap}
\usepackage{amsmath}
\usepackage{tabularx}
\usepackage{array}
\usepackage{graphicx}
\usepackage[parfill]{parskip}
\usepackage{lastpage}
\usepackage{setspace} % single spacing between lines
\usepackage{blindtext} % for generated text - can remove
\usepackage{titlesec} % set header spacing
\setlength{\parindent}{15pt} % paragraph indent

\titlespacing{\section}{0pt}{\parskip}{-\parskip}
\titlespacing{\subsection}{0pt}{\parskip}{-\parskip}
\titlespacing{\subsubsection}{0pt}{\parskip}{-\parskip}

\usepackage[numbered]{bookmark}
\clubpenalty=10000
\widowpenalty=10000

\usepackage{fancybox,fancyhdr}
\pagestyle{fancy}
\fancyhf{}
\fancyhead[C]{\small{Олимпиадное программирование (высокий уровень). Тренировка 07 \\ Летняя компьютерная школа ``КЭШ'', 6--26 августа 2016 года}}

%user-defined commands

\newcommand{\inputFile}{стандартный ввод}
\newcommand{\outputFile}{стандартный вывод}


\begin{document}

\singlespacing

\section*{Задача A. Лесенки}

\begin{tabularx}{\textwidth}{l l X}
    Имя входного файла: & \texttt{\inputFile} \\
    Имя выходного файла: & \texttt{\outputFile} \\
    Ограничение по времени: & $2$ секунды \\
    Ограничение по памяти: & $256$ мегабайт \\
\end{tabularx}

Лесенкой называется набор кубиков в один или несколько слоёв,
в котором каждый более верхний слой содержит кубиков меньше, чем нижний.

\begin{figure}[h]
    \includegraphics[scale=0.5]{ladder.jpg}
\end{figure}

Подсчитать число лесенок, которое можно построить из $N$ кубиков.

\subsection*{Формат входных данных}
На входе записано число $N$ ($1 \le N \le 100$).

\subsection*{Формат выходных данных}
Вывести искомое количество лесенок.


\subsection*{Примеры}

\texttt{
    \begin{tabularx}{0.9\textwidth}{| X | X |}
        \hline
        \multicolumn{1}{|c|}{\inputFile} & \multicolumn{1}{c|}{\outputFile} \\ \hline
        \parbox[t]{\textheight}{
            3 \\
        } & \parbox[t]{\textheight}{
            2 \\
        } \\ \hline
    \end{tabularx}
}
\newpage


\section*{Задача B. Бросание кубика}

\begin{tabularx}{\textwidth}{l l X}
    Имя входного файла: & \texttt{\inputFile} \\
    Имя выходного файла: & \texttt{\outputFile} \\
    Ограничение по времени: & $2$ секунды \\
    Ограничение по памяти: & $256$ мегабайт \\
\end{tabularx}

Кубик, грани которого помечены цифрами от 1 до 6, бросают $N$ раз.
Найти вероятность того, что сумма выпавших чисел будет равна $Q$.

\subsection*{Формат входных данных}
В первой строке находятся числа $N$ и $Q$ через пробел ($1 \le N \le 500$, $1 \le Q \le 3000$).

\subsection*{Формат выходных данных}
Выводится единственное вещественное число, которое должно отличаться от
истинного значения не более чем на 0.01 истинного значения.


\subsection*{Примеры}

\texttt{
    \begin{tabularx}{0.9\textwidth}{| X | X |}
        \hline
        \multicolumn{1}{|c|}{\inputFile} & \multicolumn{1}{c|}{\outputFile} \\ \hline
        \parbox[t]{\textheight}{
            1 6 \\
        } & \parbox[t]{\textheight}{
            0.16666 \\
        } \\ \hline
        \parbox[t]{\textheight}{
            1 7 \\
        } & \parbox[t]{\textheight}{
            0.0 \\
        } \\ \hline
        \parbox[t]{\textheight}{
            4 14 \\
        } & \parbox[t]{\textheight}{
            0.11265 \\
        } \\ \hline
        \parbox[t]{\textheight}{
            100 100 \\
        } & \parbox[t]{\textheight}{
            1.53e-78 \\
        } \\ \hline
    \end{tabularx}
}
\newpage


\end{document}

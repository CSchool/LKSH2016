\documentclass[12pt]{scrartcl}

\usepackage[
  a4paper, mag=1000,
  left=2cm, right=1cm, top=2cm, bottom=2cm, headsep=0.7cm, footskip=1.27cm
]{geometry}

\usepackage[T2A]{fontenc}
\usepackage[utf8]{inputenc}
\usepackage[english,russian]{babel}
\usepackage{cmap}
\usepackage{amsmath}
\usepackage{tabularx}
\usepackage{array}
\usepackage{graphicx}
\usepackage[parfill]{parskip}
\usepackage{lastpage}
\usepackage{setspace} % single spacing between lines
\usepackage{blindtext} % for generated text - can remove
\usepackage{titlesec} % set header spacing
\setlength{\parindent}{15pt} % paragraph indent

\titlespacing{\section}{0pt}{\parskip}{-\parskip}
\titlespacing{\subsection}{0pt}{\parskip}{-\parskip}
\titlespacing{\subsubsection}{0pt}{\parskip}{-\parskip}

\usepackage[numbered]{bookmark}
\clubpenalty=10000
\widowpenalty=10000

\usepackage{fancybox,fancyhdr}
\pagestyle{fancy}
\fancyhf{}
\fancyhead[C]{\small{Олимпиадное программирование (высокий уровень). Тренировка 01 \\ Летняя компьютерная школа ``КЭШ'', 6--26 августа 2016 года}}

%user-defined commands

\newcommand{\inputFile}{стандартный ввод}
\newcommand{\outputFile}{стандартный вывод}

\begin{document}

\singlespacing

\section*{Задача A. SMS}

\begin{tabularx}{\textwidth}{l l X}
    Имя входного файла: & \texttt{\inputFile} \\
    Имя выходного файла: & \texttt{\outputFile} \\
    Ограничение по времени: & $2$ секунды \\
    Ограничение по памяти: & $256$ мегабайт \\
\end{tabularx}

\begin{figure}[h]
    \includegraphics[scale=0.5]{mobila.png}
\end{figure}

Сообщения SMS сотового телефона MOBILA составлены из прописных латинских букв.
Если буква первая на кнопке, нужно нажать эту кнопку один раз,
чтобы добавить букву в сообщение. Если буква вторая --- нужно нажать кнопку дважды и т.д.
Так, чтобы набрать слово \texttt{SMS}, нужно нажать \\
\texttt{(PQRS)(PQRS)(PQRS)(PQRS)(MNO)(PQRS)(PQRS)(PQRS)(PQRS)}

Чтобы ввести две буквы, находящиеся на одной кнопке,
нужно между нажатиями клавиши сделать паузу.
Например, чтобы ввести сообщение \texttt{AA}, нужно нажать \\
\texttt{(ABC)(пауза)(ABC)}

Если на кнопке три буквы, то, как только такая кнопка нажата три раза,
последняя буква добавляется в сообщение немедленно, а следующие нажатия
той же кнопки относятся к следующей букве сообщения. Аналогично,
если на кнопке четыре буквы, то после четырёх нажатий в сообщение
будет добавлена последняя буква. То есть последовательность нажатий \\
\texttt{(ABC)(ABC)(ABC)(ABC)(пауза)(ABC)} \\
соответствует сообщению \texttt{CAA}.

К сожалению, сотовые телефоны этой модели давно не производятся,
и остался только один такой телефон. Он может произвольно вставлять
и игнорировать паузы во время ввода сообщения, что может привести к
некоторым изменениям в сообщениях. Например, введя \texttt{MOSCOWQUARTERFINAL},
можно получить вместо этого \texttt{OMSCMNWQTTARTERPDEINAL}.
Вы получили SMS-сообщение и знаете, что оригинальное сообщение содержало $N$ букв. Чтобы определить вероятность угадывания оригинального сообщения, найдите число возможных сообщений, которые могли превратиться в то, которое Вы получили.

\subsection*{Формат входных данных}
В первой строке задана длина оригинального сообщения $N$ ($1 \le N \le 80$).
Вторая строка содержит полученное SMS-сообщение.
Оно состоит из прописных латинских букв, его длина не превышает 80.

\subsection*{Формат выходных данных}
Вывести число сообщений из $N$ букв, которые, будучи набранными на на этом
телефоне, могут превратиться в данное сообщение.

\subsection*{Примеры}

\texttt{
    \begin{tabularx}{0.9\textwidth}{| X | X |}
        \hline
        \multicolumn{1}{|c|}{\inputFile} & \multicolumn{1}{c|}{\outputFile} \\ \hline
        \parbox[t]{\textheight}{
            4 \\
            MAMA \\
        } & \parbox[t]{\textheight}{
            1 \\
        } \\ \hline
        \parbox[t]{\textheight}{
            2 \\
            WWW \\
        } & \parbox[t]{\textheight}{
            2 \\
        } \\ \hline
        \parbox[t]{\textheight}{
            80 \\
            QUARTERFINAL \\
        } & \parbox[t]{\textheight}{
            0 \\
        } \\ \hline
    \end{tabularx}
}
\newpage

\section*{Задача B. Калькулятор}

\begin{tabularx}{\textwidth}{l l X}
    Имя входного файла: & \texttt{\inputFile} \\
    Имя выходного файла: & \texttt{\outputFile} \\
    Ограничение по времени: & $2$ секунды \\
    Ограничение по памяти: & $256$ мегабайт \\
\end{tabularx}

У исполнителя Калькулятор две команды, которым присвоены номера:
\begin{enumerate}
    \item прибавь 1
    \item умножь на 2
\end{enumerate}

Сколько есть программ, которые число 1 преобразуют в число $N$?

\subsection*{Формат входных данных}
На входе целое число $N$ ($1 \le N \le 1000$).

\subsection*{Формат выходных данных}
Целое число --- количество программ, которые число 1 преобразуют в число $N$.

\subsection*{Примеры}

\texttt{
    \begin{tabularx}{0.9\textwidth}{| X | X |}
        \hline
        \multicolumn{1}{|c|}{\inputFile} & \multicolumn{1}{c|}{\outputFile} \\ \hline
        \parbox[t]{\textheight}{
            5 \\
        } & \parbox[t]{\textheight}{
            4 \\
        } \\ \hline
    \end{tabularx}
}

\subsection*{Примечание}
Для $N = 5$ получаются такие варианты программ: \\
\begin{itemize}
    \item 1111
    \item 121
    \item 221
    \item 2111
\end{itemize}

Итого получается 4 варианта

\newpage

\section*{Задача C. Гангстеры}

\begin{tabularx}{\textwidth}{l l X}
    Имя входного файла: & \texttt{\inputFile} \\
    Имя выходного файла: & \texttt{\outputFile} \\
    Ограничение по времени: & $2$ секунды \\
    Ограничение по памяти: & $256$ мегабайт \\
\end{tabularx}

$N$ гангстеров собираются в ресторан. $i$-й гангстер приходит в момент времени
$T_i$ и имеет богатство $P_i$. Дверь ресторана имеет $K + 1$ степень открытости, они
обозначаются целыми числами из интервала $[0, K]$. Степень открытости двери
может изменяться на единицу в единицу времени, то есть дверь может открыться
на единицу, закрыться на единицу или остаться в том же состоянии. В начальный
момент времени дверь закрыта (степень открытости 0). $i$-й гангстер заходит в ресторан,
только если дверь открыта специально для него, то есть когда степень открытости
двери соответствует его полноте $S_i$. Если в момент, когда гангстер подходит к
ресторану, степень открытости двери не соответствует его полноте, он уходит
и больше не возвращается. Ресторан работает в интервале времени $[0, T]$.
Требуется собрать гангстеров с максимальным суммарным богатством в ресторане,
открывая и закрывая дверь соответствующим образом.

\subsection*{Формат входных данных}
В первой строке находятся числа $N, K, T$, во второй --- $T_1, T_2, \ldots, T_N$,
в третьей --- $P_1, P_2, \ldots, P_N$. в четвёртой --- $S_1, S_2, \ldots, S_N$.
$1 \le N \le 100$;
$1 \le K \le 100$;
$1 \le T \le 30000$;
$0 \le T_i \le T$;
$1 \le P_i \le 300$;
$1 \le S_i \le K$; все числа целые.

\subsection*{Формат выходных данных}
Вывести одно число --- максимальное суммарное богатство гангстеров,
попавших в ресторан. Если зайти не удалось никому, вывести 0.

\subsection*{Примеры}

\texttt{
    \begin{tabularx}{0.9\textwidth}{| X | X |}
        \hline
        \multicolumn{1}{|c|}{\inputFile} & \multicolumn{1}{c|}{\outputFile} \\ \hline
        \parbox[t]{\textheight}{
4 10 20 \\
10 16 8 16 \\
10 11 15 1 \\
10 7 1 8 \\
        } & \parbox[t]{\textheight}{
            26 \\
        } \\ \hline
        \parbox[t]{\textheight}{
2 17 100 \\
5 0 \\
50 33 \\
6 1 \\
        } & \parbox[t]{\textheight}{
            0 \\
        } \\ \hline
    \end{tabularx}
}

\newpage

\end{document}

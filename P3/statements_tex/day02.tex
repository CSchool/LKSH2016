\documentclass[12pt]{scrartcl}

\usepackage[
  a4paper, mag=1000,
  left=2cm, right=1cm, top=2cm, bottom=2cm, headsep=0.7cm, footskip=1.27cm
]{geometry}

\usepackage[T2A]{fontenc}
\usepackage[utf8]{inputenc}
\usepackage[english,russian]{babel}
\usepackage{cmap}
\usepackage{amsmath}
\usepackage{tabularx}
\usepackage{array}
\usepackage[parfill]{parskip}
\usepackage{lastpage}
\usepackage{setspace} % single spacing between lines
\usepackage{blindtext} % for generated text - can remove
\usepackage{titlesec} % set header spacing
\setlength{\parindent}{15pt} % paragraph indent

\titlespacing{\section}{0pt}{\parskip}{-\parskip}
\titlespacing{\subsection}{0pt}{\parskip}{-\parskip}
\titlespacing{\subsubsection}{0pt}{\parskip}{-\parskip}

\usepackage[numbered]{bookmark}
\clubpenalty=10000
\widowpenalty=10000

\usepackage{fancybox,fancyhdr}
\pagestyle{fancy}
\fancyhf{}
\fancyhead[C]{\small{Олимпиадное программирование (высокий уровень). Тренировка 02 \\ Летняя компьютерная школа ``КЭШ'', 6--26 августа 2016 года}}

%user-defined commands

\newcommand{\inputFile}{стандартный ввод}
\newcommand{\outputFile}{стандартный вывод}

\begin{document}

\singlespacing

\section*{Задача A. Разложение на числа Фибоначчи}

\begin{tabularx}{\textwidth}{l l X}
    Имя входного файла: & \texttt{\inputFile} \\
    Имя выходного файла: & \texttt{\outputFile} \\
    Ограничение по времени: & $1$ секунда \\
    Ограничение по памяти: & $256$ мегабайт \\
\end{tabularx}

Числа Фибоначчи --- элементы последовательности,
в которой первые два числа равны 1 и 1, а каждое последующее равно сумме двух предыдущих.

Вам дано целое число $N$.
Необходимо разложить его как сумму таких слагаемых, что:
\begin{itemize}
    \item Каждое слагаемое является числом Фибоначчи
    \item Никакие два слагаемых в сумме не образуют число Фибоначчи
    \item Все слагаемые попарно различны
\end{itemize}

\subsection*{Формат входных данных}
На вход подаётся целое число $N$ ($1 \le N \le 10^9$)

\subsection*{Формат выходных данных}
В первой строке вывести количество слагаемых в разложении числа $N$.
Во второй строке через пробел вывести сами слагаемые в порядке возрастания.

\subsection*{Примеры}

\texttt{
    \begin{tabularx}{0.9\textwidth}{| X | X |}
        \hline
        \multicolumn{1}{|c|}{\inputFile} & \multicolumn{1}{c|}{\outputFile} \\ \hline
        \parbox[t]{\textheight}{
            3 \\
        } & \parbox[t]{\textheight}{
            1 \\
            3 \\
        } \\ \hline
        \parbox[t]{\textheight}{
            4 \\
        } & \parbox[t]{\textheight}{
            2 \\
            1 3 \\
        } \\ \hline
        \parbox[t]{\textheight}{
            7 \\
        } & \parbox[t]{\textheight}{
            2 \\
            2 5 \\
        } \\ \hline
    \end{tabularx}
}
\newpage


\section*{Задача B. Сообщение}

\begin{tabularx}{\textwidth}{l l X}
    Имя входного файла: & \texttt{\inputFile} \\
    Имя выходного файла: & \texttt{\outputFile} \\
    Ограничение по времени: & $2$ секунды \\
    Ограничение по памяти: & $256$ мегабайт \\
\end{tabularx}

В сообщении, состоящем из одних русских букв и пробелов,
каждую букву заменили её порядковым номером в русском алфавите
(А --- 1, Б --- 2, $\ldots$, Я --- 33), а пробел --- нулём.
Требуется по заданной последовательности цифр найти количество исходных сообщений,
из которых она могла получиться.

\subsection*{Формат входных данных}
На вход подаётся последовательность цифр, её длина не превышает 100.

\subsection*{Формат выходных данных}
Вывести количество возможных исходных сообщений.

\subsection*{Примеры}

\texttt{
    \begin{tabularx}{0.9\textwidth}{| X | X |}
        \hline
        \multicolumn{1}{|c|}{\inputFile} & \multicolumn{1}{c|}{\outputFile} \\ \hline
        \parbox[t]{\textheight}{
            1025 \\
        } & \parbox[t]{\textheight}{
            4 \\
        } \\ \hline
    \end{tabularx}
}
\newpage


\section*{Задача C. Последовательность Фибоначчи}

\begin{tabularx}{\textwidth}{l l X}
    Имя входного файла: & \texttt{\inputFile} \\
    Имя выходного файла: & \texttt{\outputFile} \\
    Ограничение по времени: & $2$ секунды \\
    Ограничение по памяти: & $256$ мегабайт \\
\end{tabularx}

$F_k$ --- бесконечная последовательность целых чисел, которая удовлетворяет условию Фибоначчи:
$F_k = F_{k - 1} + F_{k - 2}$ (для любого целого $k$). Даны $i, F_i, j, F_j, n$ ($i \ne j$).
Найти $F_n$.

Пример части последовательности:

\begin{tabular}{| l | l | l | l | l | l | l | l | l | l |}
    \hline
    $k$   & -2 & -1 & 0  & 1 & 2 & 3 & 4 & 5  & 6
    \\ \hline
    $F_k$ & -5 & 4  & -1 & 3 & 2 & 5 & 7 & 12 & 19
    \\ \hline
\end{tabular}

\subsection*{Формат входных данных}
В первой строке записаны числа
$i, F_i, j, F_j, n$ ($-10^3 \le i, j, n \le 10^3, -2 \cdot 10^9 \le F_k \le 2 \cdot 10^9$
($k = min(i, j, n) \ldots max(i, j, n)$))

\subsection*{Формат выходных данных}
Вывести одно число $F_n$

\subsection*{Примеры}

\texttt{
    \begin{tabularx}{0.9\textwidth}{| X | X |}
        \hline
        \multicolumn{1}{|c|}{\inputFile} & \multicolumn{1}{c|}{\outputFile} \\ \hline
        \parbox[t]{\textheight}{
            3 5 -1 4 5 \\
        } & \parbox[t]{\textheight}{
            12 \\
        } \\ \hline
    \end{tabularx}
}
\newpage
\end{document}

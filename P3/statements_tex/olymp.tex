\documentclass[12pt]{scrartcl}

\usepackage[
  a4paper, mag=1000,
  left=2cm, right=1cm, top=2cm, bottom=2cm, headsep=0.7cm, footskip=1.27cm
]{geometry}

\usepackage[T2A]{fontenc}
\usepackage[utf8]{inputenc}
\usepackage[english,russian]{babel}
\usepackage{cmap}
\usepackage{amsmath}
\usepackage{tabularx}
\usepackage{array}
\usepackage[parfill]{parskip}
\usepackage{lastpage}
\usepackage{setspace} % single spacing between lines
\usepackage{blindtext} % for generated text - can remove
\usepackage{titlesec} % set header spacing
\setlength{\parindent}{15pt} % paragraph indent

\titlespacing{\section}{0pt}{\parskip}{-\parskip}
\titlespacing{\subsection}{0pt}{\parskip}{-\parskip}
\titlespacing{\subsubsection}{0pt}{\parskip}{-\parskip}

\usepackage[numbered]{bookmark}
\clubpenalty=10000
\widowpenalty=10000

\usepackage{fancybox,fancyhdr}
\pagestyle{fancy}
\fancyhf{}
\fancyhead[C]{\small{Олимпиадное программирование (высокий уровень). Итоговая олимпиада \\ Летняя компьютерная школа ``КЭШ'', 6--26 августа 2016 года}}

%user-defined commands

\newcommand{\inputFile}{стандартный ввод}
\newcommand{\outputFile}{стандартный вывод}

\begin{document}

\singlespacing

\section*{Задача A. Число $e$}

\begin{tabularx}{\textwidth}{l l X}
    Имя входного файла: & \texttt{\inputFile} \\
    Имя выходного файла: & \texttt{\outputFile} \\
    Ограничение по времени: & $2$ секунды \\
    Ограничение по памяти: & $256$ мегабайт \\
\end{tabularx}

Выведите в выходной файл округленное до $n$ знаков после десятичной точки число $e$.
Число $e$, округленное до 25-ти знаков после десятичной точки, равно $2.7182818284590452353602875$.

\subsection*{Формат входных данных}
Первая строка входного файла содержит целое число $n$ ($0 \le n \le 25$). 

\subsection*{Формат выходных данных}
В выходной файл выведите ответ на задачу. 

\subsection*{Примеры}

\texttt{
    \begin{tabularx}{0.9\textwidth}{| X | X |}
        \hline
        \multicolumn{1}{|c|}{\inputFile} & \multicolumn{1}{c|}{\outputFile} \\ \hline
        \parbox[t]{\textheight}{
            0
        } & \parbox[t]{\textheight}{
            3
        } \\ \hline
        \parbox[t]{\textheight}{
            25
        } & \parbox[t]{\textheight}{
            2.7182818284590452353602875
        } \\ \hline
        \parbox[t]{\textheight}{
            13
        } & \parbox[t]{\textheight}{
            2.7182818284590
        } \\ \hline
    \end{tabularx}
}
\newpage



\section*{Задача B. Цифры}

\begin{tabularx}{\textwidth}{l l X}
    Имя входного файла: & \texttt{\inputFile} \\
    Имя выходного файла: & \texttt{\outputFile} \\
    Ограничение по времени: & $2$ секунды \\
    Ограничение по памяти: & $256$ мегабайт \\
\end{tabularx}

Чтобы привлечь самых маленьких детей в летнюю компьютерную школу ``КЭШ'' руководством было
приобретено очень много красивых пластмассовых цифр. Однажды мимо коробки,
где они хранились, проходил ди-джей Герман.
Заметив среди цифр нули он решил, что это сушки, и все их съел.
Теперь детям на уроках математики приходится решать задачи без использования нулей. Например, иногда их просят составить любое число с суммой цифр равной $N$.
Мы не спрашиваем, было ли у Германа расстройство желудка.
То, что нас интересует, это сколько разных чисел с суммой разрядов $N$ можно составить из цифр от 1 до 9. 

\subsection*{Формат входных данных}
Первая строка входного файла содержит целое число $N$ ($1 \le N \le 29$). 

\subsection*{Формат выходных данных}
Единственное число --- количество чисел, которые можно составить из цифр, сумма которых равна $N$.

\subsection*{Примеры}

\texttt{
    \begin{tabularx}{0.9\textwidth}{| X | X |}
        \hline
        \multicolumn{1}{|c|}{\inputFile} & \multicolumn{1}{c|}{\outputFile} \\ \hline
        \parbox[t]{\textheight}{
            4
        } & \parbox[t]{\textheight}{
            8
        } \\ \hline
    \end{tabularx}
}

\subsection*{Примечание}
Следующие числа имеют сумму разрядов, равную 4: 1111, 112, 121, 211, 22, 13, 31, 4. 
\newpage

\section*{Задача C. Столовские котлеты}

\begin{tabularx}{\textwidth}{l l X}
    Имя входного файла: & \texttt{\inputFile} \\
    Имя выходного файла: & \texttt{\outputFile} \\
    Ограничение по времени: & $2$ секунды \\
    Ограничение по памяти: & $256$ мегабайт \\
\end{tabularx}

Главный повар решил устроить в ЛКШ День Уважения к Повару.
Для этого он приготовил школьникам $N$ необычайно вкусных котлет и втайне постановил,
что первый пожаловавший отведать поварское кушанье школьник должен получить
наибольшее количество вкусных котлет, а каждый последующий --- строго меньше,
чем предыдущий (повару очень не нравилось, когда к приготовленному им обеду опаздывали и тот вынужден был остывать).

Конечно, введенное правило оставляет существенный произвол в числе котлет,
получаемых очередным явившимся школьником, и это число не в последнюю очередь
будет зависеть от его предыдущего поведения в столовой,
а также от волшебных слов, произносимых им.
Например, 6 котлет могут быть в результате распределены по одной из следующих четырех схем:
$3 + 2 + 1$ (три котлеты первому из пришедших школьников, две - второму и одну - третьему),
$4 + 2$, $5 + 1$ и $6$ (все котлеты съедает счастливчик, пришедший первым). 

Напишите программу, определяющую, каким количеством различных способов повар может
распределить приготовленное лакомство среди школьников. 

\subsection*{Формат входных данных}
Входной файл содержит одно целое число $N$ --- количество приготовленных поваром котлет ($0 \le N \le 200$).

\subsection*{Формат выходных данных}
Выходной файл должен содержать одно целое число, равное количеству возможных распределений котлет.

\subsection*{Примеры}

\texttt{
    \begin{tabularx}{0.9\textwidth}{| X | X |}
        \hline
        \multicolumn{1}{|c|}{\inputFile} & \multicolumn{1}{c|}{\outputFile} \\ \hline
        \parbox[t]{\textheight}{
            6
        } & \parbox[t]{\textheight}{
            4
        } \\ \hline
    \end{tabularx}
}

\newpage


\section*{Задача D. Вестник ``КЭШ''}

\begin{tabularx}{\textwidth}{l l X}
    Имя входного файла: & \texttt{\inputFile} \\
    Имя выходного файла: & \texttt{\outputFile} \\
    Ограничение по времени: & $2$ секунды \\
    Ограничение по памяти: & $256$ мегабайт \\
\end{tabularx}

Однажды отряду \textnumero3 в летней компьютерной школе ``КЭШ'' поручили написать газету о событиях в лагере.
Текст им составить удалось, однако на оформление газеты сил не хватило, так как дело было после изнурительной
утренней зарядки. К счастью, к ним в руки попала газета отряда \textnumero1.
Теперь третий отряд хочет вычеркнуть из найденной газеты некоторые символы,
чтобы получить свой текст. Помогите им определить, смогут ли они это сделать. 

\subsection*{Формат входных данных}
Входные данные состоят из двух строк. В первой строке --- текст газеты первого отряда,
во второй --- текст газеты третьего отряда.
Длина обоих текстов не меньше 1 и не превышает 30000 символов.
Каждый пробел в любой из строк учитывается как отдельный символ. Заглавные и строчные буквы различаются. 

\subsection*{Формат выходных данных}
Выведите \texttt{YES}, если третьему отряду удастся оформить газету и \texttt{NO} в противном случае. 

\subsection*{Примеры}

\texttt{
    \begin{tabularx}{0.9\textwidth}{| X | X |}
        \hline
        \multicolumn{1}{|c|}{\inputFile} & \multicolumn{1}{c|}{\outputFile} \\ \hline
        \parbox[t]{\textheight}{
            NOT A TEXT FOR THINKING ON \\
            NOTHING \\ 
        } & \parbox[t]{\textheight}{
            YES \\
        } \\ \hline
    \end{tabularx}
}

\newpage



\section*{Задача E. Операционные системы}

\begin{tabularx}{\textwidth}{l l X}
    Имя входного файла: & \texttt{\inputFile} \\
    Имя выходного файла: & \texttt{\outputFile} \\
    Ограничение по времени: & $2$ секунды \\
    Ограничение по памяти: & $256$ мегабайт \\
\end{tabularx}

Васин жесткий диск состоит из $M$ секторов.
Вася последовательно устанавливал на него различные операционные системы следующим методом:
он создавал новый раздел диска из последовательных секторов, начиная с сектора номер $a_i$ и до сектора $b_i$
включительно, и устанавливал на него очередную систему.
При этом если очередной раздел хотя бы по одному сектору пересекается с каким-то ранее созданным разделом,
то ранее созданный раздел ``затирается'',
и операционная система, которая на него была установлена, больше не может быть загружена.

Напишите программу, которая по информации о том, какие разделы на диске создавал Вася, определит,
сколько в итоге работающих операционных систем установлено и в настоящий момент работает на Васином компьютере. 

\subsection*{Формат входных данных}
Сначала вводятся натуральное число $M$ --- количество секторов на жестком диске ($1 \le M \le 10^9$)
и целое число $N$ --- количество разделов, которое последовательно создавал Вася ($0 \le N \le 1000$).

Далее идут $N$ пар чисел $a_i$ и $b_i$, задающих номера
начального и конечного секторов раздела ($1 \le a_i \le b_i \le M$). 

\subsection*{Формат выходных данных}
Выведите одно число --- количество работающих операционных систем на Васином компьютере. 

\subsection*{Примеры}

\texttt{
    \begin{tabularx}{0.9\textwidth}{| X | X |}
        \hline
        \multicolumn{1}{|c|}{\inputFile} & \multicolumn{1}{c|}{\outputFile} \\ \hline
        \parbox[t]{\textheight}{
10 \\
3 \\
1 3 \\
4 7 \\
3 4  \\
        } & \parbox[t]{\textheight}{
            1 \\
        } \\ \hline
        \parbox[t]{\textheight}{
10 \\
4 \\
1 3 \\
4 5 \\
7 8 \\
4 6 \\
        } & \parbox[t]{\textheight}{
            3 \\
        } \\ \hline
    \end{tabularx}
}
\newpage


\section*{Задача F. Робот}

\begin{tabularx}{\textwidth}{l l X}
    Имя входного файла: & \texttt{\inputFile} \\
    Имя выходного файла: & \texttt{\outputFile} \\
    Ограничение по времени: & $2$ секунды \\
    Ограничение по памяти: & $256$ мегабайт \\
\end{tabularx}

Робот Р-2008-2009 предназначен для исследования просторов Флатландии,
которые, как известно, представляют собой части плоскости, разбитые на единичные квадраты (клетки)
вертикальными и горизонтальными прямыми.
Программы для этого робота достаточно просты, так как написаны на языке программирования,
который содержит всего четыре команды:

\begin{enumerate}
    \item сдвинуться на клетку вверх --- \texttt{U}
    \item сдвинуться на клетку вниз --- \texttt{D}
    \item сдвинуться на клетку влево --- \texttt{L}
    \item сдвинуться на клетку вправо --- \texttt{R}
\end{enumerate}

Ваша задача состоит в написании программы, которая будет анализировать некоторые свойства программы для
робота Р-2008-2009, --- а именно, предположим, что исследуемая область представляет собой бесконечную
во всех четырех направлениях плоскость. Задана программа для робота Р-2008-2009.
Необходимо найти число клеток плоскости, которые он посетит более одного раза.
Заметим, что это число не зависит от того, в какой клетке изначально находится робот. 

\subsection*{Формат входных данных}
Первая строка входного файла содержит программу для робота.
Она состоит только из символов \texttt{U}, \texttt{D}, \texttt{L}, \texttt{R}.
Ее длина положительна и не превосходит 1000 символов. 

\subsection*{Формат выходных данных}
В выходной файл выведите ответ на задачу.

\subsection*{Примеры}

\texttt{
    \begin{tabularx}{0.9\textwidth}{| X | X |}
        \hline
        \multicolumn{1}{|c|}{\inputFile} & \multicolumn{1}{c|}{\outputFile} \\ \hline
        \parbox[t]{\textheight}{
            ULDR 
        } & \parbox[t]{\textheight}{
            1 
        } \\ \hline
        \parbox[t]{\textheight}{
            URLD 
        } & \parbox[t]{\textheight}{
            2 
        } \\ \hline
    \end{tabularx}
}
\newpage



\section*{Задача G. Сдача}

\begin{tabularx}{\textwidth}{l l X}
    Имя входного файла: & \texttt{\inputFile} \\
    Имя выходного файла: & \texttt{\outputFile} \\
    Ограничение по времени: & $2$ секунды \\
    Ограничение по памяти: & $256$ мегабайт \\
\end{tabularx}

Когда Аня и Света покупали подарок, возникла интересная ситуация.
У них была в распоряжении только одна большая купюра, а у продавца --- некоторое количество мелочи.
Дело происходило утром, поэтому продавцу нужно было экономить мелочь,
и он хотел отдать сдачу минимальным количеством монет. Подумав некоторое время, они точно определили,
с каким количеством монет продавцу придется расстаться. А вы сможете решить такую задачу? 

\subsection*{Формат входных данных}
В первой строке записано число $n$ ($1 \le n \le 10$) --- количество различных номиналов монет,
содержащихся в кассе. Можно считать, что количество монет каждого номинала достаточно.

На следующей строке содержится $n$ целых чисел $a_i$ ($0 < a_i \le 2000$) --- номиналы монет.
В третьей строке записано одно число $k$ ($1 \le k \le 10^6$) --- сумма, которую нужно набрать. 

\subsection*{Формат выходных данных}
Выведите минимальное количество монет, которое придется отдать продавцу, или -1,
если продавец вообще не сможет дать им сдачу. 

\subsection*{Примеры}

\texttt{
    \begin{tabularx}{0.9\textwidth}{| X | X |}
        \hline
        \multicolumn{1}{|c|}{\inputFile} & \multicolumn{1}{c|}{\outputFile} \\ \hline
        \parbox[t]{\textheight}{
            3 \\
            1 3 5 \\
            13 \\
        } & \parbox[t]{\textheight}{
            3
        } \\ \hline
        \parbox[t]{\textheight}{
            4 \\
            5 6 7 8 \\
            9 \\
        } & \parbox[t]{\textheight}{
            -1 
        } \\ \hline
    \end{tabularx}
}
\newpage

\end{document}

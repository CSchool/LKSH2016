\documentclass[12pt]{scrartcl}

\usepackage[
  a4paper, mag=1000,
  left=2cm, right=1cm, top=2cm, bottom=2cm, headsep=0.7cm, footskip=1.27cm
]{geometry}

\usepackage[T2A]{fontenc}
\usepackage[utf8]{inputenc}
\usepackage[english,russian]{babel}
\usepackage{cmap}
\usepackage{amsmath}
\usepackage{tabularx}
\usepackage{array}
\usepackage[parfill]{parskip}
\usepackage{lastpage}
\usepackage{setspace} % single spacing between lines
\usepackage{blindtext} % for generated text - can remove
\usepackage{titlesec} % set header spacing
\setlength{\parindent}{15pt} % paragraph indent

\titlespacing{\section}{0pt}{\parskip}{-\parskip}
\titlespacing{\subsection}{0pt}{\parskip}{-\parskip}
\titlespacing{\subsubsection}{0pt}{\parskip}{-\parskip}

\usepackage[numbered]{bookmark}
\clubpenalty=10000
\widowpenalty=10000

\usepackage{fancybox,fancyhdr}
\pagestyle{fancy}
\fancyhf{}
\fancyhead[C]{\small{Олимпиадное программирование (высокий уровень). Тренировка 03 \\ Летняя компьютерная школа ``КЭШ'', 6--26 августа 2016 года}}

%user-defined commands

\newcommand{\inputFile}{стандартный ввод}
\newcommand{\outputFile}{стандартный вывод}

\begin{document}

\singlespacing

\section*{Задача A. Размен}

\begin{tabularx}{\textwidth}{l l X}
    Имя входного файла: & \texttt{\inputFile} \\
    Имя выходного файла: & \texttt{\outputFile} \\
    Ограничение по времени: & $2$ секунды \\
    Ограничение по памяти: & $256$ мегабайт \\
\end{tabularx}

В этой задаче нужно подсчитать количество способов разменять $N$ центов,
используя монеты достоинством 1, 5, 10, 25 и 50 центов.

\subsection*{Формат входных данных}
На вход подаётся одно число $N$ ($1 \le N \le 10^5$)

\subsection*{Формат выходных данных}
Вывести одно число --- ответ на задачу

\subsection*{Примеры}

\texttt{
    \begin{tabularx}{0.9\textwidth}{| X | X |}
        \hline
        \multicolumn{1}{|c|}{\inputFile} & \multicolumn{1}{c|}{\outputFile} \\ \hline
        \parbox[t]{\textheight}{
            6 \\
        } & \parbox[t]{\textheight}{
            2 \\
        } \\ \hline
        \parbox[t]{\textheight}{
            15 \\
        } & \parbox[t]{\textheight}{
            6 \\
        } \\ \hline
    \end{tabularx}
}
\newpage

\section*{Задача B. Копилка}

\begin{tabularx}{\textwidth}{l l X}
    Имя входного файла: & \texttt{\inputFile} \\
    Имя выходного файла: & \texttt{\outputFile} \\
    Ограничение по времени: & $2$ секунды \\
    Ограничение по памяти: & $256$ мегабайт \\
\end{tabularx}

Задан вес $E$ пустой копилки и вес $F$ копилки с монетами.
В копилке могут находиться монеты $N$ видов, для каждого вида известна ценность $P_i$ и вес $W_i$ одной монеты.
Найти минимальную и максимальную сумму денег, которые могут находиться в копилке.

\subsection*{Формат входных данных}
В первой строке находятся числа $E$ и $F$, во второй --- число $N$.
В следующих $N$ строках --- по два числа, $P_i$ и $W_i$.

$1 \le E \le F \le 10^4$; $1 \le N \le 500$; $1 \le P_i \le 5 \cdot 10^4$; $1 \le W_i \le 10^4$

\subsection*{Формат выходных данных}
Выводятся два числа через пробел --- минимальная и максимальная суммы.
Если копилка не может иметь точно заданный вес при условии,
что она наполнена монетами заданных видов, --- вывести \texttt{"This is impossible."} (без кавычек).

\subsection*{Примеры}

\texttt{
    \begin{tabularx}{0.9\textwidth}{| X | X |}
        \hline
        \multicolumn{1}{|c|}{\inputFile} & \multicolumn{1}{c|}{\outputFile} \\ \hline
        \parbox[t]{\textheight}{
            1000 1100 \\
            2 \\
            1 1 \\
            5 2 \\
        } & \parbox[t]{\textheight}{
            100 250 \\
        } \\ \hline
        \parbox[t]{\textheight}{
            1000 1010 \\
            2 \\
            6 3 \\
            2 2 \\
        } & \parbox[t]{\textheight}{
            10 16 \\
        } \\ \hline
        \parbox[t]{\textheight}{
            1000 2000 \\
            1 \\
            10 3 \\
        } & \parbox[t]{\textheight}{
            This is impossible. \\
        } \\ \hline
    \end{tabularx}
}
\newpage
\end{document}

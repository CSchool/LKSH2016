\documentclass[12pt]{scrartcl}

\usepackage[
  a4paper, mag=1000,
  left=2cm, right=1cm, top=2cm, bottom=2cm, headsep=0.7cm, footskip=1.27cm
]{geometry}

\usepackage[T2A]{fontenc}
\usepackage[utf8]{inputenc}
\usepackage[english,russian]{babel}
\usepackage{cmap}
\usepackage{amsmath}
\usepackage{tabularx}
\usepackage{array}
\usepackage[parfill]{parskip}
\usepackage{lastpage}
\usepackage{setspace} % single spacing between lines
\usepackage{blindtext} % for generated text - can remove
\usepackage{titlesec} % set header spacing
\setlength{\parindent}{15pt} % paragraph indent

\titlespacing{\section}{0pt}{\parskip}{-\parskip}
\titlespacing{\subsection}{0pt}{\parskip}{-\parskip}
\titlespacing{\subsubsection}{0pt}{\parskip}{-\parskip}

\usepackage[numbered]{bookmark}
\clubpenalty=10000
\widowpenalty=10000

\usepackage{fancybox,fancyhdr}
\pagestyle{fancy}
\fancyhf{}
\fancyhead[C]{\small{Олимпиадное программирование (высокий уровень). Тренировка 06 \\ Летняя компьютерная школа ``КЭШ'', 6--26 августа 2016 года}}

%user-defined commands

\newcommand{\inputFile}{стандартный ввод}
\newcommand{\outputFile}{стандартный вывод}

\begin{document}

\singlespacing

\section*{Задача A. Маршрут}

\begin{tabularx}{\textwidth}{l l X}
    Имя входного файла: & \texttt{\inputFile} \\
    Имя выходного файла: & \texttt{\outputFile} \\
    Ограничение по времени: & $2$ секунды \\
    Ограничение по памяти: & $256$ мегабайт \\
\end{tabularx}

В таблице из $N$ строк и $N$ столбцов клетки заполнены цифрами от 0 до 9.
Требуется найти такой путь из клетки (1, 1) в клетку ($N$, $N$),
чтобы сумма цифр в клетках, через которые он пролегает, была минимальной;
из любой клетки ходить можно только вниз или вправо.

\subsection*{Формат входных данных}
В первой строке находится число $N$. В следующих $N$ строках содержатся по $N$ цифр без пробелов.
$2 \le N \le 250$

\subsection*{Формат выходных данных}
Выводятся $N$ строк по $N$ символов.
Символ решётка показывает, что маршрут проходит через эту клетку,
а минус --- что не проходит. Если путей с минимальной суммой цифр несколько, вывести любой.


\subsection*{Примеры}

\texttt{
    \begin{tabularx}{0.9\textwidth}{| X | X |}
        \hline
        \multicolumn{1}{|c|}{\inputFile} & \multicolumn{1}{c|}{\outputFile} \\ \hline
        \parbox[t]{\textheight}{
3 \\
943 \\
216 \\
091 \\
        } & \parbox[t]{\textheight}{
\#-{}- \\
\#\#\# \\ 
-{}-\# \\
        } \\ \hline
    \end{tabularx}
}
\newpage


\section*{Задача B. Минимальный путь в таблице}

\begin{tabularx}{\textwidth}{l l X}
    Имя входного файла: & \texttt{\inputFile} \\
    Имя выходного файла: & \texttt{\outputFile} \\
    Ограничение по времени: & $2$ секунды \\
    Ограничение по памяти: & $256$ мегабайт \\
\end{tabularx}

В прямоугольной таблице $N\times{}M$ (в каждой клетке которой записано некоторое число)
в начале игрок находится в левой верхней клетке.
За один ход ему разрешается перемещаться в соседнюю клетку либо вправо,
либо вниз (влево и вверх перемещаться запрещено).
При проходе через клетку с игрока берут столько у.е.,
какое число записано в этой клетке (деньги берут также за первую и последнюю клетки его пути).
Требуется найти минимальную сумму у.е., заплатив которую игрок может попасть в правый нижний угол.

\subsection*{Формат входных данных}
Во входном файле задано два числа $N$ и $M$ --- размеры таблицы
($1 \le N, M \le 20$). Затем идет $N$ строк по $M$ чисел в каждой --- размеры штрафов в у.е.
за прохождение через соответствующие клетки (числа от 0 до 100).

\subsection*{Формат выходных данных}
В выходной файл запишите минимальную сумму, потратив которую можно попасть в правый нижний угол.

\subsection*{Примеры}

\texttt{
    \begin{tabularx}{0.9\textwidth}{| X | X |}
        \hline
        \multicolumn{1}{|c|}{\inputFile} & \multicolumn{1}{c|}{\outputFile} \\ \hline
        \parbox[t]{\textheight}{
3 4 \\
1 1 1 1 \\
5 2 2 100 \\
9 4 2 1 \\
        } & \parbox[t]{\textheight}{
            8 \\
        } \\ \hline
    \end{tabularx}
}
\newpage



\section*{Задача C. Маршрут 2}

\begin{tabularx}{\textwidth}{l l X}
    Имя входного файла: & \texttt{\inputFile} \\
    Имя выходного файла: & \texttt{\outputFile} \\
    Ограничение по времени: & $2$ секунды \\
    Ограничение по памяти: & $256$ мегабайт \\
\end{tabularx}

Дана матрица $N\times{}N$, заполненная положительными числами.
Путь по матрице начинается в левом верхнем углу.
За один ход можно пройти в соседнюю по вертикали или горизонтали клетку (если она существует).
Нельзя ходить по диагонали, нельзя оставаться на месте. Требуется найти максимальную сумму чисел,
стоящих в клетках по пути длиной $K$ (клетку можно посещать несколько раз).

\subsection*{Формат входных данных}
В первой строке находятся разделенные пробелом числа $N$ и $K$. Затем идут $N$ строк по $N$ чисел в каждой.
$2 \le N \le 100$, элементы матрицы имеют значения от 1 до 9999, $1 \le K \le 2000$, все числа целые.

\subsection*{Формат выходных данных}
Вывести одно число -- максимальную сумму.

\subsection*{Примеры}

\texttt{
    \begin{tabularx}{0.9\textwidth}{| X | X |}
        \hline
        \multicolumn{1}{|c|}{\inputFile} & \multicolumn{1}{c|}{\outputFile} \\ \hline
        \parbox[t]{\textheight}{
5 7 \\
1 1 1 1 1 \\
1 1 3 1 9 \\
1 1 6 1 1 \\
1 1 3 1 1 \\
1 1 1 1 1 \\
        } & \parbox[t]{\textheight}{
            21 \\
        } \\ \hline
    \end{tabularx}
}
\newpage
\end{document}

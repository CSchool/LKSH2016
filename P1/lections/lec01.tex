\documentclass[10pt,a4paper]{article}
\usepackage[T2A]{fontenc}
\usepackage[utf8]{inputenc}
\usepackage{amsmath}
\usepackage{amsfonts}
\usepackage{amssymb}
\usepackage{graphicx}

\usepackage[russian, english]{babel}

\usepackage{color}
\usepackage{listings}

\definecolor{cmntClr}{rgb}{0.1,0.5,0.2}
\definecolor{keyWClr}{rgb}{0.2,0.1,0.5}
\definecolor{emphClr}{rgb}{0.5,0.1,0.2}
\definecolor{bkgrndClr}{rgb}{0.95,0.95,0.95}

\begin{document}

\section*{Краткое напоминание}

\lstset{
    language=C++,
    commentstyle=\color{cmntClr},
    keywordstyle=\color{keyWClr},
    emphstyle=\color{emphClr},
    numbers=left,
    basicstyle=\ttfamily,
    backgroundcolor=\color{bkgrndClr},
    frame=single
}

\subsection*{Простейшая программа}

\begin{lstlisting}
#include <iostream>

using namespace std;

int main() {
    return 0; // test
}
\end{lstlisting}

Точка начала исполнения программы -- функция \lstinline|main()|. Т.к. внутри неё сейчас нет никаких других инструкций, то программа ничего не делает. На начальном этапе, весь ваш код, будет располагаться внутри функции \lstinline|main()|.

\subsection*{Переменные и встроенные типы}

Часто в программе нужно хранить какие-то данные, например числа.
Для этого используются переменные. Переменные -- это именнованный
кусочек памяти. 

Как объявить переменные:

\begin{lstlisting}
<type_name> <variable_name> 
\end{lstlisting}

Вы не можете использовать не объявленные ранее переменные.

Какого типа можно создавать переменные? Есть слдующие встроенные типы:

\begin{enumerate}
\item Для целых числел: \lstinline|int|, \lstinline|long long| --- отличаются только диапазоном допустимых значений.
\item Для символов: \lstinline|char|
\item Логический: \lstinline|bool|
\item Вещественные числа: \lstinline|float|, \lstinline|double|
\end{enumerate}

Как записать что-то в переменную?

\begin{lstlisting}
<variable_name> = <exspression>
\end{lstlisting}

Пример:

\begin{lstlisting}
int a = 20;
int b = 1;
int c = 2 * (a + b); // 42
\end{lstlisting}

\subsubsection*{Целые числа}

Целые числа можно:

\begin{enumerate}
\item складывать \lstinline|a + b|
\item умножать \lstinline|a * b|
\item нацелочисленное деление \lstinline|a / b|
\item брать остаток от деления \lstinline|a % b|
\end{enumerate}

\subsection*{Вывод на экран}

Чтобы вывести что-то нужно написать:

\begin{lstlisting}
cout << expression1 << expression2;
\end{lstlisting}

т.е. выведем что-нибудь ещё:

\begin{lstlisting}
int a = 42;
cout << "Ultimate question's answer" << " " << a; 
// output: Ultimate question's answer 42
\end{lstlisting}

\subsection*{Считывание}

Считываение выглядит аналогично выводу:

\begin{lstlisting}
cin >> variable1 >> variable1;
\end{lstlisting}

\end{document}
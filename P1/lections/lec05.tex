\documentclass[10pt,a4paper]{article}
\usepackage[T2A]{fontenc}
\usepackage[utf8]{inputenc}
\usepackage{amsmath}
\usepackage{amsfonts}
\usepackage{amssymb}
\usepackage{graphicx}

\usepackage[russian, english]{babel}

\usepackage{color}
\usepackage{listings}

\definecolor{cmntClr}{rgb}{0.1,0.5,0.2}
\definecolor{keyWClr}{rgb}{0.2,0.1,0.5}
\definecolor{emphClr}{rgb}{0.5,0.1,0.2}
\definecolor{bkgrndClr}{rgb}{0.95,0.95,0.95}

\begin{document}

\parindent=0cm

\section*{Краткое напоминание}

\lstset{
    language=C++,
    commentstyle=\color{cmntClr},
    keywordstyle=\color{keyWClr},
    emphstyle=\color{emphClr},
    numbers=left,
    basicstyle=\ttfamily,
    backgroundcolor=\color{bkgrndClr},
    frame=single
}

\subsection*{Циклы}

Некоторые действия хочется повторять несколько раз. Для таких
целей существуют циклы. Мы с вами рассмотрим цикл \lstinline|while|

\textit{Пример:}
\begin{lstlisting}
while (expression) {
    // block A
}
\end{lstlisting}

Блок \textit{A} будет выполняться, пока \lstinline|expression|
имеет значение \lstinline|true|.

Из цикла можно выйти раньше если будет выполнена инструкция \lstinline|break|


Простой пример цикла, который выводит числа от 1 до 10:

\begin{lstlisting}
int cnt = 1;
while (cnt <= 10) {
    cout << cnt << '\n';
    cnt = cnt + 1;
}
\end{lstlisting}

Переменная \lstinline|cnt| --- выполняет роль счётчика. Пока она не больше 10, мы её выводим на экран.

\end{document}
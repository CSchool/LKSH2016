\documentclass[10pt,a4paper]{article}
\usepackage[T2A]{fontenc}
\usepackage[utf8]{inputenc}
\usepackage{amsmath}
\usepackage{amsfonts}
\usepackage{amssymb}
\usepackage{graphicx}

\usepackage[russian, english]{babel}

\usepackage{color}
\usepackage{listings}

\definecolor{cmntClr}{rgb}{0.1,0.5,0.2}
\definecolor{keyWClr}{rgb}{0.2,0.1,0.5}
\definecolor{emphClr}{rgb}{0.5,0.1,0.2}
\definecolor{bkgrndClr}{rgb}{0.95,0.95,0.95}

\begin{document}

\section*{Краткое напоминание}

\lstset{
    language=C++,
    commentstyle=\color{cmntClr},
    keywordstyle=\color{keyWClr},
    emphstyle=\color{emphClr},
    numbers=left,
    basicstyle=\ttfamily,
    backgroundcolor=\color{bkgrndClr},
    frame=single
}

\subsection*{Ветвления}

Иногда нужно уметь выполнять некоторые действия при выполнении
каких-то условий. Для таких ситуаций есть ветвления.


Пример:
\begin{lstlisting}
if (expression) {
    // block A
} else {
    // block B
}
\end{lstlisting}


Блок \textit{A} выполнится, если \lstinline|expression| имеет 
значение \lstinline|true|, в противном случае выполнится блок \textit{B}.
Часть с \lstinline|else| необязательная.

Как может выглядеть выражение возвращающее логические значения?
Это может быть сравнение:

\begin{enumerate}
\item \lstinline|==| --- равенство
\item \lstinline|!=| --- неравенство
\item \lstinline|>| --- меньше
\item \lstinline|<| --- больше
\item \lstinline|>=| --- больше или равно
\item \lstinline|<=| --- меньше или равно
\end{enumerate}

Либо коомбинация логических выражений соединённых через:

\begin{enumerate}
\item \lstinline|!| --- логическое НЕ
\item \lstinline|&&| --- логическое И
\item || --- логическое ИЛИ
\end{enumerate}

Пример программы, выводящей ОК, если введённое число положительное
и меньше 10.

\begin{lstlisting}
#include <iostream>

using namespace std;

int main() {
    int a;
    cin >> a;
    if (0 < a && a < 10) {
        cout << "OK";
    }
}
\end{lstlisting}

\end{document}
\documentclass[12pt]{scrartcl}

\usepackage[
  a4paper, mag=1000,
  left=2cm, right=1cm, top=2cm, bottom=2cm, headsep=0.7cm, footskip=1.27cm
]{geometry}

\usepackage[T2A]{fontenc}
\usepackage[utf8]{inputenc}
\usepackage[english,russian]{babel}
\usepackage{cmap}
\usepackage{amsmath}
\usepackage{tabularx}
\usepackage{array}
\usepackage[parfill]{parskip}
\usepackage{tabularx}
\usepackage{lastpage}

\usepackage[numbered]{bookmark}
\clubpenalty=10000
\widowpenalty=10000

\usepackage{fancybox,fancyhdr}
\pagestyle{fancy}
\fancyhf{}
\fancyhead[C]{\small{Начальное программирование. Тренировка 01 \\ Летняя компьютерная школа ``КЭШ'', 6--26 августа 2016 года}}

\begin{document}

\section*{Задача A. A + B}

\begin{tabularx}{\textwidth}{l l X}
    Имя входного файла: & \texttt{stdin} \\
    Имя выходного файла: & \texttt{stdout} \\
    Ограничение по времени: & $2$ секунды \\
    Ограничение по памяти: & $256$ мегабайт \\
\end{tabularx}

На вход даны два целых числа, найти сумму.

\subsection*{Формат входных данных}

Два целых числа, разделённые пробелом.

\subsection*{Формат выходных данных}

Одно целое число, сумма исходных.

\subsection*{Примеры}

\texttt{
    \begin{tabularx}{\textwidth}{| X | X |}
        \hline
        stdin & stdout \\ \hline
        40 2 & 42 \\ \hline
    \end{tabularx}
}
\newpage

\section*{Задача B. Четность}

\begin{tabularx}{\textwidth}{l l X}
    Имя входного файла: & \texttt{stdin} \\
    Имя выходного файла: & \texttt{stdout} \\
    Ограничение по времени: & $2$ секунды \\
    Ограничение по памяти: & $256$ мегабайт \\
\end{tabularx}

Дано число, определить чётность.

\subsection*{Формат входных данных}

Одно целое число.

\subsection*{Формат выходных данных}

Вывести 1, если число чётное, 0 иначе.

\subsection*{Примеры}

\texttt{
    \begin{tabularx}{\textwidth}{| X | X |}
        \hline
        stdin & stdout \\ \hline
        42 & 1 \\ \hline
    \end{tabularx}
}
\newpage

\section*{Задача C. Cумма цифр}

\begin{tabularx}{\textwidth}{l l X}
    Имя входного файла: & \texttt{stdin} \\
    Имя выходного файла: & \texttt{stdout} \\
    Ограничение по времени: & $2$ секунды \\
    Ограничение по памяти: & $256$ мегабайт \\
\end{tabularx}

Дано число, найти сумму его цифр.

\subsection*{Формат входных данных}

Одно целое $0 \leq n < 1000$

\subsection*{Формат выходных данных}

Одно целое число -- сумма цифр.

\subsection*{Примеры}

\texttt{
    \begin{tabularx}{\textwidth}{| X | X |}
        \hline
        stdin & stdout \\ \hline
        42 & 6 \\ \hline
    \end{tabularx}
}
\newpage

\section*{Задача D. Swap}

\begin{tabularx}{\textwidth}{l l X}
    Имя входного файла: & \texttt{stdin} \\
    Имя выходного файла: & \texttt{stdout} \\
    Ограничение по времени: & $2$ секунды \\
    Ограничение по памяти: & $256$ мегабайт \\
\end{tabularx}

Даны 2 числа. Прочтите их в 2 переменные и обменяйте значения переменных местами. Проверяется исходный код.

\subsection*{Примеры}

\texttt{
    \begin{tabularx}{\textwidth}{| X | X |}
        \hline
        stdin & stdout \\ \hline
        2 4 & 4 2 \\ \hline
    \end{tabularx}
}

\newpage

\section*{Задача E. Number}

\begin{tabularx}{\textwidth}{l l X}
    Имя входного файла: & \texttt{stdin} \\
    Имя выходного файла: & \texttt{stdout} \\
    Ограничение по времени: & $2$ секунды \\
    Ограничение по памяти: & $256$ мегабайт \\
\end{tabularx}

Даны 4 числа. Вывести одно число, без ведущих, в котором цифры, соответствуют входным числам.

\subsection*{Формат входных данных}

4 целых числа.

\subsection*{Формат выходных данных}

Одно целое число -- ответ на задачу.

\subsection*{Примеры}

\texttt{
    \begin{tabularx}{\textwidth}{| X | X |}
        \hline
        stdin & stdout \\ \hline
        0 0 4 2 & 42 \\ \hline
    \end{tabularx}
}

\newpage

\section*{Задача F. Hello, world!}

\begin{tabularx}{\textwidth}{l l X}
    Имя входного файла: & \texttt{stdin} \\
    Имя выходного файла: & \texttt{stdout} \\
    Ограничение по времени: & $2$ секунды \\
    Ограничение по памяти: & $256$ мегабайт \\
\end{tabularx}

Вывести "Hello, world!".

\subsection*{Формат выходных данных}

Одно целое число -- ответ на задачу.

\subsection*{Примеры}

\texttt{
    \begin{tabularx}{\textwidth}{| X | X |}
        \hline
        stdin & stdout \\ \hline
        & Hello, world! \\ \hline
    \end{tabularx}
}

\end{document}

\documentclass[12pt]{scrartcl}

\usepackage[
  a4paper, mag=1000,
  left=2cm, right=1cm, top=2cm, bottom=2cm, headsep=0.7cm, footskip=1.27cm
]{geometry}

\usepackage[T2A]{fontenc}
\usepackage[utf8]{inputenc}
\usepackage[english,russian]{babel}
\usepackage{cmap}
\usepackage{amsmath}
\usepackage{tabularx}
\usepackage{array}
\usepackage[parfill]{parskip}
\usepackage{tabularx}
\usepackage{lastpage}

\usepackage[numbered]{bookmark}
\clubpenalty=10000
\widowpenalty=10000

\usepackage{graphicx}

\usepackage{fancybox,fancyhdr}
\pagestyle{fancy}
\fancyhf{}
\fancyhead[C]{\small{Начальное программирование. Тренировка 04 \\ Летняя компьютерная школа ``КЭШ'', 6--26 августа 2016 года}}

\begin{document}

\section*{Задача A. Реверс числа}

\begin{tabularx}{\textwidth}{l l X}
    Имя входного файла: & \texttt{stdin} \\
    Имя выходного файла: & \texttt{stdout} \\
    Ограничение по времени: & $2$ секунды \\
    Ограничение по памяти: & $256$ мегабайт \\
\end{tabularx}

Дано 4х значное число. Нужно вывести ``перевёрнутое'' число.

\subsection*{Формат входных данных}

Целое число из 4х знаков.

\subsection*{Формат выходных данных}

Перевёрнутое число.

\subsection*{Примеры}

\texttt{
    \begin{tabularx}{\textwidth}{| X | X |}
        \hline
        stdin & stdout \\ \hline
        4021 & 1204 \\ \hline
    \end{tabularx}
}
\newpage

\section*{Задача B. Квадрат}

\begin{tabularx}{\textwidth}{l l X}
    Имя входного файла: & \texttt{stdin} \\
    Имя выходного файла: & \texttt{stdout} \\
    Ограничение по времени: & $2$ секунды \\
    Ограничение по памяти: & $256$ мегабайт \\
\end{tabularx}

Квадрат расстояния между двумя точками $a$ и $b$ на плоскости вычисляется
по формуле: 

$$(a_x - b_x)^2 + (a_y - b_y)^2$$

Вам даны координаты точек, найти квадрат расстояния между ними.

\subsection*{Формат входных данных}

4 целых числа через пробел: $a_x$, $a_y$, $b_x$, $b_y$.

\subsection*{Формат выходных данных}

Одно целое число -- квадрат расстояния между двумя точками.

\subsection*{Примеры}

\texttt{
    \begin{tabularx}{\textwidth}{| X | X |}
        \hline
        stdin & stdout \\ \hline
        0 0 1 1 & 2 \\ \hline
    \end{tabularx}
}
\newpage

\section*{Задача C. Площадь квадрата}

\begin{tabularx}{\textwidth}{l l X}
    Имя входного файла: & \texttt{stdin} \\
    Имя выходного файла: & \texttt{stdout} \\
    Ограничение по времени: & $2$ секунды \\
    Ограничение по памяти: & $256$ мегабайт \\
\end{tabularx}

Дана длина диагонали квадрата. Найти площадь квадрата.

\textit{Подсказака}:

\includegraphics[scale=0.5]{img.jpg}

\subsection*{Формат входных данных}

Одно натуральное число --- диагональ квадрата.

\subsection*{Формат выходных данных}

Площадь квадрата.

\subsection*{Примеры}

\texttt{
    \begin{tabularx}{\textwidth}{| X | X |}
        \hline
        stdin & stdout \\ \hline
        2 & 2 \\ \hline
    \end{tabularx}
}
\newpage

\section*{Задача D. Произведение}

\begin{tabularx}{\textwidth}{l l X}
    Имя входного файла: & \texttt{stdin} \\
    Имя выходного файла: & \texttt{stdout} \\
    Ограничение по времени: & $2$ секунды \\
    Ограничение по памяти: & $256$ мегабайт \\
\end{tabularx}

Дано трёхзначное число. Найти произведение его чисел.

\subsection*{Формат входных данных}

Трёхзначное число.

\subsection*{Формат выходных данных}

Одно число -- произведение цифр.

\subsection*{Примеры}

\texttt{
    \begin{tabularx}{\textwidth}{| X | X |}
        \hline
        stdin & stdout \\ \hline
        241 & 8 \\ \hline
    \end{tabularx}
}

\newpage

\end{document}

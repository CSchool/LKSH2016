\documentclass[12pt]{scrartcl}

\usepackage[
  a4paper, mag=1000,
  left=2cm, right=1cm, top=2cm, bottom=2cm, headsep=0.7cm, footskip=1.27cm
]{geometry}

\usepackage[T2A]{fontenc}
\usepackage[utf8]{inputenc}
\usepackage[english,russian]{babel}
\usepackage{cmap}
\usepackage{amsmath}
\usepackage{tabularx}
\usepackage{array}
\usepackage[parfill]{parskip}
\usepackage{tabularx}
\usepackage{lastpage}

\usepackage[numbered]{bookmark}
\clubpenalty=10000
\widowpenalty=10000

\usepackage{fancybox,fancyhdr}
\pagestyle{fancy}
\fancyhf{}
\fancyhead[C]{\small{Начальное программирование.\\ Летняя компьютерная школа ``КЭШ'', 6--26 августа 2016 года}}

\begin{document}

\section*{Задача A. Ингридиент}

\begin{tabularx}{\textwidth}{l l X}
    Имя входного файла: & \texttt{stdin} \\
    Имя выходного файла: & \texttt{stdout} \\
    Ограничение по времени: & $2$ секунды \\
    Ограничение по памяти: & $256$ мегабайт \\
\end{tabularx}

Пожалуй, самым важным секретом семьи Дживсов является рецепт бальзама
против головной боли. Главный ингридиет до сих пор остаётся неизвестным. 
Однако его количество спрятано, как сумма квадратов цифр номера на бирке 
твидового пиджака. Найти сумму квадратов цифр четырёхзначного числа.

\subsection*{Формат входных данных}

Номер бирки твидового пиджака --- целое число не больше 10000.

\subsection*{Формат выходных данных}

Количество секретного ингридиента --- сумма квадратов цифр входного числа.

\subsection*{Примеры}

\texttt{
    \begin{tabularx}{\textwidth}{| X | X |}
        \hline
        stdin & stdout \\ \hline
        1234 & 30 \\ \hline
    \end{tabularx}
}
\newpage

\section*{Задача B. Гольф}

\begin{tabularx}{\textwidth}{l l X}
    Имя входного файла: & \texttt{stdin} \\
    Имя выходного файла: & \texttt{stdout} \\
    Ограничение по времени: & $2$ секунды \\
    Ограничение по памяти: & $256$ мегабайт \\
\end{tabularx}

Друг юности Бертрама Вустера Бинго любитель игры в гольф. Всем известно,
что мячик летит по параболе. Бинго хочет отправить мячик в полёт как можно дальше, 
желательно не разбив окна местной оранжереи. Вам свыше дано
уравнение движения мячика, проверьте попадёт ли Бинго в окна оранжереи.

\subsection*{Формат входных данных}

На первой строчке три числа --- коэффиценты уравнения $ax^2 + bx + c = 0$.
На второй --- координата оранжереи.

\subsection*{Формат выходных данных}

Вывести \texttt{YES} --- если попадёт и \texttt{NO} иначе.

\subsection*{Комментарий к условию}

Бинго может попасть в оранжерею, только если её координата является корнем уравнения траектории.

\subsection*{Примеры}

\texttt{
    \begin{tabularx}{\textwidth}{| X | X |}
        \hline
        stdin & stdout \\ \hline
        \parbox[t]{\textheight}{
                            -1 0 4  \\
                            2 \\
                        } & YES \\ \hline
    \end{tabularx}
}
\newpage

\section*{Задача C. Встреча}

\begin{tabularx}{\textwidth}{l l X}
    Имя входного файла: & \texttt{stdin} \\
    Имя выходного файла: & \texttt{stdout} \\
    Ограничение по времени: & $2$ секунды \\
    Ограничение по памяти: & $256$ мегабайт \\
\end{tabularx}

Господин Вустер желает организовать встречу нескольких людей на
мосту. Он разослал приглашения, но Дживс выражает сомнения, что
план выполним. Помогите проверить Бертраму, смогут ли все
приглашённые люди пересечься на мосту?

\subsection*{Формат входных данных}

На первой строке --- число $n$, далее $n$ строчек, в каждой по 
два числа: время прибытия очередного гостя на мост и его ухода
в абсолютных величинах (т.е. в виде целого числа)

\subsection*{Формат выходных данных}

Вывести \texttt{YES} --- если все гости смогут встретиться, иначе
\texttt{NO}.

\subsection*{Примеры}

\texttt{
    \begin{tabularx}{\textwidth}{| X | X |}
        \hline
        stdin & stdout \\ \hline
        \parbox[t]{\textheight}{
        3 \\
        1 10 \\
        2 8 \\
        5 9 \\} & YES \\ \hline
    \end{tabularx}
}
\newpage


\section*{Задача D. Жалование Фибоначчи}

\begin{tabularx}{\textwidth}{l l X}
    Имя входного файла: & \texttt{stdin} \\
    Имя выходного файла: & \texttt{stdout} \\
    Ограничение по времени: & $2$ секунды \\
    Ограничение по памяти: & $256$ мегабайт \\
\end{tabularx}

Добрые люди посоветовали Бертраму Вустеру выплачивать месячное жалование Дживсу
как $i$-ое число Фибоначчи. Дживс очень винимателен к финансовому благополучию
своего хозяина и ему интересно, как скоро Бертрам Вустер разорится. Вам известен
месячный доход Бертрама, вывести номер месяца, когда он не сможет выплатить доход
своему камердинеру.

\textit{Числа фибоначчи это последовательнотсь, получаемая по следующей формуле:
$a_{n+2} = a_{n+1} + a_n$, где $a_2 = 1$ и $a_1 = 1$.}


\subsection*{Формат входных данных}

Одно число --- месячный доход сэра Вустера. 

\subsection*{Формат выходных данных}

Одно число --- номер месяца, когда Дживс не сможет получить жалование.

\subsection*{Примеры}

\texttt{
    \begin{tabularx}{\textwidth}{| X | X |}
        \hline
        stdin & stdout \\ \hline
        7132 & 20 \\ \hline
    \end{tabularx}
}

\newpage

\section*{Задача E. Тарелочки}

\begin{tabularx}{\textwidth}{l l X}
    Имя входного файла: & \texttt{stdin} \\
    Имя выходного файла: & \texttt{stdout} \\
    Ограничение по времени: & $2$ секунды \\
    Ограничение по памяти: & $256$ мегабайт \\
\end{tabularx}

Как все уже догадались, Дживс держит дом в порядке, но полка с 
тарелками --- это особое место. Там тарелочки образуют правильную 
скобочную последовательность. Дживс ушёл за продуктами, а вы остались
дома и разбили парочку тарелок. Теперь вы в спешке хотите проверить,
сохранилась ли правильная скобочная последовательность или нет. 
Поторопитесь, вы уже слышите шаги на лестнице.

\subsection*{Формат входных данных}

На первой строке число $n$, на второй $n$ скобочек, разделённых пробелом.

\subsection*{Формат выходных данных}

Вывести \texttt{YES}, если последовательность правильная и \texttt{NO}.

\subsection*{Комментарий к условию}

Для считывания символов используйте переменные типа \texttt{char}.

\subsection*{Примеры}

\texttt{
    \begin{tabularx}{\textwidth}{| X | X |}
        \hline
        stdin & stdout \\ \hline
        \parbox[t]{\textheight}{
                            5 \\
                            ( ) ( ) ( \\
                        } & NO \\ \hline
    \end{tabularx}
}

\end{document}

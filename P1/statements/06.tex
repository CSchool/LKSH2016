\documentclass[12pt]{scrartcl}

\usepackage[
  a4paper, mag=1000,
  left=2cm, right=1cm, top=2cm, bottom=2cm, headsep=0.7cm, footskip=1.27cm
]{geometry}

\usepackage[T2A]{fontenc}
\usepackage[utf8]{inputenc}
\usepackage[english,russian]{babel}
\usepackage{cmap}
\usepackage{amsmath}
\usepackage{tabularx}
\usepackage{array}
\usepackage[parfill]{parskip}
\usepackage{tabularx}
\usepackage{lastpage}

\usepackage[numbered]{bookmark}
\clubpenalty=10000
\widowpenalty=10000

\usepackage{fancybox,fancyhdr}
\pagestyle{fancy}
\fancyhf{}
\fancyhead[C]{\small{Начальное программирование. Тренировка 06 \\ Летняя компьютерная школа ``КЭШ'', 6--26 августа 2016 года}}

\begin{document}

\section*{Задача A. Квадрат}

\begin{tabularx}{\textwidth}{l l X}
    Имя входного файла: & \texttt{stdin} \\
    Имя выходного файла: & \texttt{stdout} \\
    Ограничение по времени: & $2$ секунды \\
    Ограничение по памяти: & $256$ мегабайт \\
\end{tabularx}

Дано число $n$. Вывести квадрат из чисел (закономерность смотри в пример).

\subsection*{Формат входных данных}

Одно число $n$.

\subsection*{Формат выходных данных}

Квадрат из чисел.

\subsection*{Примеры}

\texttt{
    \begin{tabularx}{\textwidth}{| X | X |}
        \hline
        stdin & stdout \\ \hline
        3 & \parbox[t]{\textheight}{
                            1 2 3 \\
                            4 5 6 \\
                            7 8 9 \\
                        } \\ \hline
    \end{tabularx}
}
\newpage

\section*{Задача B. Нечётные}

\begin{tabularx}{\textwidth}{l l X}
    Имя входного файла: & \texttt{stdin} \\
    Имя выходного файла: & \texttt{stdout} \\
    Ограничение по времени: & $2$ секунды \\
    Ограничение по памяти: & $256$ мегабайт \\
\end{tabularx}

Даны $n$ чисел. Вывести количество нечётных.

\subsection*{Формат входных данных}

На первой строчке, число $n$, на второй $n$ чисел.

\subsection*{Формат выходных данных}

Одно число --- количество нечётных.

\subsection*{Примеры}

\texttt{
    \begin{tabularx}{\textwidth}{| X | X |}
        \hline
        stdin & stdout \\ \hline
        \parbox[t]{\textheight}{
                            5 \\
                            1 2 3 4 5 \\
                        } & 3 \\ \hline
    \end{tabularx}
}
\newpage

\section*{Задача C. Факториал}

\begin{tabularx}{\textwidth}{l l X}
    Имя входного файла: & \texttt{stdin} \\
    Имя выходного файла: & \texttt{stdout} \\
    Ограничение по времени: & $2$ секунды \\
    Ограничение по памяти: & $256$ мегабайт \\
\end{tabularx}

Факториалом числа $n$ является произведение чисел от 1 до $n$.

\subsection*{Формат входных данных}

Число $n$.

\subsection*{Формат выходных данных}

Факториал числа $n$.

\subsection*{Примеры}

\texttt{
    \begin{tabularx}{\textwidth}{| X | X |}
        \hline
        stdin & stdout \\ \hline
        4 & 24 \\ \hline
    \end{tabularx}
}
\newpage


\section*{Задача D. Количество вхождений}

\begin{tabularx}{\textwidth}{l l X}
    Имя входного файла: & \texttt{stdin} \\
    Имя выходного файла: & \texttt{stdout} \\
    Ограничение по времени: & $2$ секунды \\
    Ограничение по памяти: & $256$ мегабайт \\
\end{tabularx}

Даны $n$ чисел и чило $a$, вывести число вхождений числа $a$.
\subsection*{Формат входных данных}

На первой строке числа $n$ и $a$, на второй $n$ чисел.

\subsection*{Формат выходных данных}

Одно число --- число вхождений числа $a$.

\subsection*{Примеры}

\texttt{
    \begin{tabularx}{\textwidth}{| X | X |}
        \hline
        stdin & stdout \\ \hline
        \parbox[t]{\textheight}{
                            5 2 \\
                            1 2 3 2 5 \\
                        } & 2 \\ \hline
    \end{tabularx}
}

\newpage

\section*{Задача E. Сумма}

\begin{tabularx}{\textwidth}{l l X}
    Имя входного файла: & \texttt{stdin} \\
    Имя выходного файла: & \texttt{stdout} \\
    Ограничение по времени: & $2$ секунды \\
    Ограничение по памяти: & $256$ мегабайт \\
\end{tabularx}

Даны $n$ чисел, вывести сумму.
\subsection*{Формат входных данных}

На первой строке число $n$, на второй $n$ чисел.

\subsection*{Формат выходных данных}

Одно число --- сумма.

\subsection*{Примеры}

\texttt{
    \begin{tabularx}{\textwidth}{| X | X |}
        \hline
        stdin & stdout \\ \hline
        \parbox[t]{\textheight}{
                            5 \\
                            1 2 3 2 5 \\
                        } & 13 \\ \hline
    \end{tabularx}
}


\end{document}

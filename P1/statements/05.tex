\documentclass[12pt]{scrartcl}

\usepackage[
  a4paper, mag=1000,
  left=2cm, right=1cm, top=2cm, bottom=2cm, headsep=0.7cm, footskip=1.27cm
]{geometry}

\usepackage[T2A]{fontenc}
\usepackage[utf8]{inputenc}
\usepackage[english,russian]{babel}
\usepackage{cmap}
\usepackage{amsmath}
\usepackage{tabularx}
\usepackage{array}
\usepackage[parfill]{parskip}
\usepackage{tabularx}
\usepackage{lastpage}

\usepackage[numbered]{bookmark}
\clubpenalty=10000
\widowpenalty=10000

\usepackage{fancybox,fancyhdr}
\pagestyle{fancy}
\fancyhf{}
\fancyhead[C]{\small{Начальное программирование. Тренировка 05 \\ Летняя компьютерная школа ``КЭШ'', 6--26 августа 2016 года}}

\begin{document}

\section*{Задача A. Сумма цифр - 2}

\begin{tabularx}{\textwidth}{l l X}
    Имя входного файла: & \texttt{stdin} \\
    Имя выходного файла: & \texttt{stdout} \\
    Ограничение по времени: & $2$ секунды \\
    Ограничение по памяти: & $256$ мегабайт \\
\end{tabularx}

Дано число. Найти сумму его цифр.

\subsection*{Формат входных данных}

Одно целое число.

\subsection*{Формат выходных данных}

Вывести сумму цифр.

\subsection*{Примеры}

\texttt{
    \begin{tabularx}{\textwidth}{| X | X |}
        \hline
        stdin & stdout \\ \hline
        12345 & 15 \\ \hline
    \end{tabularx}
}
\newpage

\section*{Задача B. Возведение в степень}

\begin{tabularx}{\textwidth}{l l X}
    Имя входного файла: & \texttt{stdin} \\
    Имя выходного файла: & \texttt{stdout} \\
    Ограничение по времени: & $2$ секунды \\
    Ограничение по памяти: & $256$ мегабайт \\
\end{tabularx}

Даны два числа: $x$ и $k$ найти $x^k$

\subsection*{Формат входных данных}

Два целых числа.

\subsection*{Формат выходных данных}

Одно число --- степень.

\subsection*{Примеры}

\texttt{
    \begin{tabularx}{\textwidth}{| X | X |}
        \hline
        stdin & stdout \\ \hline
        2 3 & 8 \\ \hline
    \end{tabularx}
}
\newpage

\section*{Задача C. Логарифмирование}

\begin{tabularx}{\textwidth}{l l X}
    Имя входного файла: & \texttt{stdin} \\
    Имя выходного файла: & \texttt{stdout} \\
    Ограничение по времени: & $2$ секунды \\
    Ограничение по памяти: & $256$ мегабайт \\
\end{tabularx}

Дано два числа: $x$  и $y$. Найти такое $k$ что $x^k = y$

\subsection*{Формат входных данных}

Два числа $x$ и $y$.

\subsection*{Формат выходных данных}

Одно число -- ответ.

\subsection*{Примеры}

\texttt{
    \begin{tabularx}{\textwidth}{| X | X |}
        \hline
        stdin & stdout \\ \hline
        2 8 & 3 \\ \hline
    \end{tabularx}
}
\newpage


\section*{Задача D. Простые числа}

\begin{tabularx}{\textwidth}{l l X}
    Имя входного файла: & \texttt{stdin} \\
    Имя выходного файла: & \texttt{stdout} \\
    Ограничение по времени: & $2$ секунды \\
    Ограничение по памяти: & $256$ мегабайт \\
\end{tabularx}

Проверить число на простоту.

\subsection*{Формат входных данных}

Одно число $x$.

\subsection*{Формат выходных данных}

Вывести \texttt{YES}, если число простое, \texttt{NO} --- иначе.

\subsection*{Примеры}

\texttt{
    \begin{tabularx}{\textwidth}{| X | X |}
        \hline
        stdin & stdout \\ \hline
        7 & YES \\ \hline
    \end{tabularx}
}

\newpage

\section*{Задача E. Калькулятор}

\begin{tabularx}{\textwidth}{l l X}
    Имя входного файла: & \texttt{stdin} \\
    Имя выходного файла: & \texttt{stdout} \\
    Ограничение по времени: & $2$ секунды \\
    Ограничение по памяти: & $256$ мегабайт \\
\end{tabularx}

Дана формула, нужно её посчитать.

\subsection*{Формат входных данных}

Формула числа и операции разделены пробелом. Допустимые операции
сложение и вычитание. Признак конца формулы \texttt{!}.

\subsection*{Формат выходных данных}

Посчитанная формула.

\subsection*{Примеры}

\texttt{
    \begin{tabularx}{\textwidth}{| X | X |}
        \hline
        stdin & stdout \\ \hline
        7 + 1 - 2 + 3 - 8 !& 1 \\ \hline
    \end{tabularx}
}

\newpage

\end{document}

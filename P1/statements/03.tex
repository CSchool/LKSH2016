\documentclass[12pt]{scrartcl}

\usepackage[
  a4paper, mag=1000,
  left=2cm, right=1cm, top=2cm, bottom=2cm, headsep=0.7cm, footskip=1.27cm
]{geometry}

\usepackage[T2A]{fontenc}
\usepackage[utf8]{inputenc}
\usepackage[english,russian]{babel}
\usepackage{cmap}
\usepackage{amsmath}
\usepackage{tabularx}
\usepackage{array}
\usepackage[parfill]{parskip}
\usepackage{tabularx}
\usepackage{lastpage}

\usepackage[numbered]{bookmark}
\clubpenalty=10000
\widowpenalty=10000

\usepackage{fancybox,fancyhdr}
\pagestyle{fancy}
\fancyhf{}
\fancyhead[C]{\small{Начальное программирование. Тренировка 03 \\ Летняя компьютерная школа ``КЭШ'', 6--26 августа 2016 года}}

\begin{document}

\section*{Задача A. Египетский треугольник}

\begin{tabularx}{\textwidth}{l l X}
    Имя входного файла: & \texttt{stdin} \\
    Имя выходного файла: & \texttt{stdout} \\
    Ограничение по времени: & $2$ секунды \\
    Ограничение по памяти: & $256$ мегабайт \\
\end{tabularx}

Треугольник называется египетским, если длины его сторон --- целые числа
и для них выполняется условие теоремы пифагора:

$$ a^2 + b^2 = c^2 $$

Даны длины сторон. Проверить является ли треугольник египетским.

\subsection*{Формат входных данных}

Три целых числа разделённых пробелом.

\subsection*{Формат выходных данных}

\texttt{YES} --- если треугольник египетский, \texttt{NO} --- иначе.

\subsection*{Примеры}

\texttt{
    \begin{tabularx}{\textwidth}{| X | X |}
        \hline
        stdin & stdout \\ \hline
        3 4 5 & YES \\ \hline
    \end{tabularx}
}
\newpage

\section*{Задача B. Days}

\begin{tabularx}{\textwidth}{l l X}
    Имя входного файла: & \texttt{stdin} \\
    Имя выходного файла: & \texttt{stdout} \\
    Ограничение по времени: & $2$ секунды \\
    Ограничение по памяти: & $256$ мегабайт \\
\end{tabularx}

На вход подаётся номер дня недели. Вывести название дня.

\subsection*{Формат входных данных}

Число от 1 до 7.

\subsection*{Формат выходных данных}

Сокращённое название дня недели: Mon, Tue, Wed, Thu, Fri, Sat, Sun.

\subsection*{Примеры}

\texttt{
    \begin{tabularx}{\textwidth}{| X | X |}
        \hline
        stdin & stdout \\ \hline
        2 & Wed \\ \hline
    \end{tabularx}
}
\newpage

\section*{Задача C. Проверить число на палиндром}

\begin{tabularx}{\textwidth}{l l X}
    Имя входного файла: & \texttt{stdin} \\
    Имя выходного файла: & \texttt{stdout} \\
    Ограничение по времени: & $2$ секунды \\
    Ограничение по памяти: & $256$ мегабайт \\
\end{tabularx}

Дано 4х значное число. Проверить, является ли оно палиндромом.

\subsection*{Формат входных данных}

4х значное число.

\subsection*{Формат выходных данных}

\texttt{YES} --- если число палиндром, \texttt{NO} --- иначе.

\subsection*{Примеры}

\texttt{
    \begin{tabularx}{\textwidth}{| X | X |}
        \hline
        stdin & stdout \\ \hline
        1221 & YES \\ \hline
    \end{tabularx}
}
\newpage


\section*{Задача D. Счастливый билет}

\begin{tabularx}{\textwidth}{l l X}
    Имя входного файла: & \texttt{stdin} \\
    Имя выходного файла: & \texttt{stdout} \\
    Ограничение по времени: & $2$ секунды \\
    Ограничение по памяти: & $256$ мегабайт \\
\end{tabularx}

Дан номер билета. Билет считается счастливым если сумма первых 3х цифр
равна сумме последних 3х. Проверить, что билет счастливый.

\subsection*{Формат входных данных}

6 чисел через пробел.

\subsection*{Формат выходных данных}

Вывести \texttt{YES}, билет счастливый, \texttt{NO} --- иначе.

\subsection*{Примеры}

\texttt{
    \begin{tabularx}{\textwidth}{| X | X |}
        \hline
        stdin & stdout \\ \hline
        0 0 0 1 1 1 & NO \\ \hline
    \end{tabularx}
}

\newpage

\end{document}

\documentclass[12pt]{scrartcl}

\usepackage[
  a4paper, mag=1000,
  left=2cm, right=1cm, top=2cm, bottom=2cm, headsep=0.7cm, footskip=1.27cm
]{geometry}

\usepackage[T2A]{fontenc}
\usepackage[utf8]{inputenc}
\usepackage[english,russian]{babel}
\usepackage{cmap}
\usepackage{amsmath}
\usepackage{tabularx}
\usepackage{array}
\usepackage[parfill]{parskip}
\usepackage{tabularx}
\usepackage{lastpage}

\usepackage[numbered]{bookmark}
\clubpenalty=10000
\widowpenalty=10000

\usepackage{fancybox,fancyhdr}
\pagestyle{fancy}
\fancyhf{}
\fancyhead[C]{\small{Начальное программирование. Тренировка 02 \\ Летняя компьютерная школа ``КЭШ'', 6--26 августа 2016 года}}

\begin{document}

\section*{Задача A. >, <, =}

\begin{tabularx}{\textwidth}{l l X}
    Имя входного файла: & \texttt{stdin} \\
    Имя выходного файла: & \texttt{stdout} \\
    Ограничение по времени: & $2$ секунды \\
    Ограничение по памяти: & $256$ мегабайт \\
\end{tabularx}

На вход даны два целых числа, проверить их на больше,
меньше или равно

\subsection*{Формат входных данных}

Два целых числа, разделённые пробелом.

\subsection*{Формат выходных данных}

Три строчки содержащие \texttt{YES} или \texttt{NO},
первая строчка для $<$, вторая для $>$, третья для $=$

\subsection*{Примеры}

\texttt{
    \begin{tabularx}{\textwidth}{| X | X |}
        \hline
        stdin & stdout \\ \hline
        40 2 & \parbox[t]{\textheight}{
                    NO \\
                    YES \\
                    NO \\
                }\\
        \hline
    \end{tabularx}
}
\newpage

\section*{Задача B. Diff}

\begin{tabularx}{\textwidth}{l l X}
    Имя входного файла: & \texttt{stdin} \\
    Имя выходного файла: & \texttt{stdout} \\
    Ограничение по времени: & $2$ секунды \\
    Ограничение по памяти: & $256$ мегабайт \\
\end{tabularx}

Дано 3 числа, вывести число различных.

\subsection*{Формат входных данных}

Три числа через пробел.

\subsection*{Формат выходных данных}

Одно число --- количество различных чисел.

\subsection*{Примеры}

\texttt{
    \begin{tabularx}{\textwidth}{| X | X |}
        \hline
        stdin & stdout \\ \hline
        0 4 2 & 3 \\ \hline
    \end{tabularx}
}
\newpage

\section*{Задача C. Min}

\begin{tabularx}{\textwidth}{l l X}
    Имя входного файла: & \texttt{stdin} \\
    Имя выходного файла: & \texttt{stdout} \\
    Ограничение по времени: & $2$ секунды \\
    Ограничение по памяти: & $256$ мегабайт \\
\end{tabularx}

Даны 3 числа, вывести минимальное из них.

\subsection*{Формат входных данных}

Три числа через пробел.

\subsection*{Формат выходных данных}

Минимальное число.

\subsection*{Примеры}

\texttt{
    \begin{tabularx}{\textwidth}{| X | X |}
        \hline
        stdin & stdout \\ \hline
        4 2 0 & 0 \\ \hline
    \end{tabularx}
}
\newpage

\section*{Задача D. Sort}

\begin{tabularx}{\textwidth}{l l X}
    Имя входного файла: & \texttt{stdin} \\
    Имя выходного файла: & \texttt{stdout} \\
    Ограничение по времени: & $2$ секунды \\
    Ограничение по памяти: & $256$ мегабайт \\
\end{tabularx}

Дано 3 числа, отсортировать по возрастанию.

\subsection*{Формат входных данных}

Три числа через пробел.

\subsection*{Формат выходных данных}

Три числа через пробел, расположенные по возрастанию.

\subsection*{Примеры}

\texttt{
    \begin{tabularx}{\textwidth}{| X | X |}
        \hline
        stdin & stdout \\ \hline
        2 4 0 & 0 2 4 \\ \hline
    \end{tabularx}
}

\newpage

\section*{Задача E. Прямоугольник}

\begin{tabularx}{\textwidth}{l l X}
    Имя входного файла: & \texttt{stdin} \\
    Имя выходного файла: & \texttt{stdout} \\
    Ограничение по времени: & $2$ секунды \\
    Ограничение по памяти: & $256$ мегабайт \\
\end{tabularx}

Дан прямоугольник со сторонами параллельными осям координат
и точка, проверить, что точка внутри прямоугольника.

\textit{Прямоугольник замкнутый, т.е. включает в себя границы.}

\subsection*{Формат входных данных}

Два числа --- левый нижний угол, два числа --- верхний правый угол, два числа --- координаты точки.

\subsection*{Формат выходных данных}

Вывести \texttt{YES}, если точка внутри, \texttt{NO} --- иначе.

\subsection*{Примеры}

\texttt{
    \begin{tabularx}{\textwidth}{| X | X |}
        \hline
        stdin & stdout \\ \hline
        0 0 1 1 1 1 & YES \\ \hline
    \end{tabularx}
}

\newpage

\end{document}
